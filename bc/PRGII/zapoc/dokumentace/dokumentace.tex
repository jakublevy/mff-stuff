% !TeX spellcheck = cs_CZ
\documentclass[11pt,a4paper,x11names]{article}
\usepackage[utf8]{inputenc}
\usepackage[czech]{babel}
\usepackage[left=1in,top=1in,right=1in,bottom=1in]{geometry}
\usepackage{caption}
\usepackage{subcaption}
\usepackage{float}
\usepackage{newfloat}
\DeclareFloatingEnvironment[name={Tabulka}]{sheetfigure}
\renewcommand{\listsheetfigurename}{Seznam tabulek}
\DeclareFloatingEnvironment[name={Ukázka kódu}]{codefigure}
\renewcommand{\listcodefigurename}{Seznam ukázek kódu}
\usepackage{enumitem}
\usepackage{tikzsymbols}
\usepackage{fancyvrb}

\usepackage{tikz}
\usepackage{tikz-qtree}
\usepackage{latexsym}
\usepackage[figure,table]{totalcount}
\usepackage{csquotes}
\MakeOuterQuote{"}
\DeclareCaptionLabelFormat{blank}{}
\usepackage{hyperref}
\hypersetup{
	colorlinks,
	citecolor=black,
	filecolor=black,
	linkcolor=black,
	urlcolor=black
}



\expandafter\def\expandafter\UrlBreaks\expandafter{\UrlBreaks%  save the current one
	\do\a\do\b\do\c\do\d\do\e\do\f\do\g\do\h\do\i\do\j%
	\do\k\do\l\do\m\do\n\do\o\do\p\do\q\do\r\do\s\do\t%
	\do\u\do\v\do\w\do\x\do\y\do\z\do\A\do\B\do\C\do\D%
	\do\E\do\F\do\G\do\H\do\I\do\J\do\K\do\L\do\M\do\N%
	\do\O\do\P\do\Q\do\R\do\S\do\T\do\U\do\V\do\W\do\X%
	\do\Y\do\Z}

\newcommand{\Csharp}{%
	{\settoheight{\dimen0}{C}C\kern-.05em \resizebox{!}{\dimen0}{\raisebox{\depth}{\#}}}}

\date{červen/červenec 2018}




\usepackage[edges]{forest}

\definecolor{foldercolor}{RGB}{124,166,198}

\tikzset{pics/folder/.style={code={%
			\node[inner sep=0pt, minimum size=#1](-foldericon){};
			\node[folder style, inner sep=0pt, minimum width=0.3*#1, minimum height=0.6*#1, above right, xshift=0.05*#1] at (-foldericon.west){};
			\node[folder style, inner sep=0pt, minimum size=#1] at (-foldericon.center){};}
	},
	pics/folder/.default={20pt},
	folder style/.style={draw=foldercolor!80!black,top color=foldercolor!40,bottom color=foldercolor}
}

\forestset{is file/.style={edge path'/.expanded={%
			([xshift=\forestregister{folder indent}]!u.parent anchor) |- (.child anchor)},
		inner sep=1pt},
	this folder size/.style={edge path'/.expanded={%
			([xshift=\forestregister{folder indent}]!u.parent anchor) |- (.child anchor) pic[solid]{folder=#1}}, inner xsep=0.6*#1},
	folder tree indent/.style={before computing xy={l=#1}},
	folder icons/.style={folder, this folder size=#1, folder tree indent=3*#1},
	folder icons/.default={12pt},
}



\usepackage{graphicx}
\graphicspath{{images/}}
\author{Jakub Levý}
\title{Dokumentace k zápočtovému programu}
\begin{document}
\null  % Empty line
\nointerlineskip  % No skip for prev line
\vfill
\let\snewpage \newpage
\let\newpage \relax
\maketitle
\let \newpage \snewpage
\vfill 
\thispagestyle{empty}
\begin{abstract}
	Tento program reimplementuje hru Space Invaders. Je napsán v jazyce \Csharp, využívá knihovnu WinForms a GDI+. Hraní hry je doprovázeno původními zvukovými efekty\break a animacemi při eliminaci nepřátel. Součástí programu je několik cheatů, pomocí kterých je možné jednodušeji testovat hru.
\end{abstract}
\pagebreak
\tableofcontents
\pagebreak
\section{O Space Invaders}
\label{sec:about_space_invaders}
Jedná se o arkádovou hru vytvořenou Tomohirem Nishikadem. Hra byla uvolněna již v roce 1978. Hráč pohybuje tankem s laserem a jeho cílem je eliminovat nepřátelskou invazi blížící se k tanku. Vzhledem ke stáří hry se jedná o jednu z prvních stříleček.
\begin{figure}[H]
	\centering
	\begin{subfigure}{.5\textwidth}
		\centering
		\includegraphics[width=.7\linewidth]{space_invaders_original.png}
		\caption{původní hra}
	\end{subfigure}%
	\begin{subfigure}{.5\textwidth}
		\centering
		\includegraphics[width=1.0\linewidth]{space_invaders_remake.png}
		\caption{moje verze}
	\end{subfigure}
	\caption{porovnání originální hry s mojí verzí}
\end{figure}
\section{Využité technologie}
Program, které se tato dokumentace týká, by nemohl vzniknout bez různých technologií,\break o kterých se rozepíšu v této kapitole.
\subsection{C\#}
Celý program byl napsán v programovacím jazyku \Csharp. Tento programovací jazyk jsem zvolil\newline z důvodu jeho jednoduchosti, výkonnosti a možnosti objektově orientovaného programování.
\subsection{WinForms}
Pro jednoduchou práci s tlačítky jsem se rozhodl použít grafické rozhraní.
%Pro jednoduché zadání vstupu a jeho případné editace jsem se rozhodl využít grafického rozhraní.
Ačkoliv knihovna WPF je modernější, zvolil jsem knihovnu WinForms ze dvou důvodů:
\begin{enumerate}
	\item kompatibilita s jinými operačními systémy než Windows,
	\item jednoduchá tvorba uživatelského rozhraní.
\end{enumerate}
\subsection{Visual Studio 2017}
U volby vývojového prostředí jsem neváhal. Konkurenční JetBrains Rider a MonoDevelop neobsahují designer pro tvorbu uživatelského prostředí aplikací využívajících knihovnu WinForms.
\subsection{GDI+}
Kvůli jednoduchosti se o vykreslování stará součást API OS Windows. Pro účely tohoto programu je dostačující.
\subsection{Adobe Photoshop}
Už jenom tento dokument obsahuje \totalfigures\ obrázků. Několik dalších je součástí programu. Často bylo potřeba udělat přesný výběr objektu, odstranit pozadí a přidat nějaký prvek do obrázku. Jedná se sice o základní funkcionality, ale samotná možnost rozdělit si části obrázku do vrstev mi značně ulehčila práci.
\section{Program}
\label{sec:program}
V této kapitole popíšu jednotlivé části programu ze dvou různých pohledů. Nejprve z pohledu uživatelského, pokračovat budu programátorskou částí, kde rozeberu vybrané bloky kódu. %\\ \\
%Nadále budu rozumět pojmem \textit{hra} tento zápočtový program.
\subsection{Uživatelská část}
V této části se, jak již název podkapitoly napovídá, zaměřím na části programu z takového pohledu, že i obyčejnému uživateli mohou přijít informace získané přečtením zajímavé a poučné, nacházejí se zde i informace, které se dají považovat za nutné ke hraní hry.
\subsubsection{Ovládání}
Pro ovládání tanku nám stačí 3 tlačítka viz následující tabulka.
\begin{sheetfigure}[H]
	\centering
\begin{tabular}{ l | c  r }
	pohyb tanku doleva & $\leftarrow$ \\ \hline
	pohyb tanku doprava & $\rightarrow$ \\ \hline
	střelba & \texttt{mezerník}, alt. \texttt{x} \\
\end{tabular}
\caption{tlačítka ovládající tank}
\end{sheetfigure}
\paragraph{Cheaty} 
Hra obsahuje celkem 5 podvodů, které nám pomůžou k poražení nepřátel. Opět jejich přehled v tabulce.
\begin{sheetfigure}[H]
	\centering
	\begin{tabular}{ l | c | r }
		\textit{popis cheatu} & \textit{od slova} & \textit{klávesová zkratka} \\ \hline
		přídání jednoho životu & \textbf{h}ealth & \texttt{h} \\ \hline
		regenerace bariéry & \textbf{b}arrier & \texttt{b} \\ \hline
		zakázání/povolení střílení nepřátel & \textbf{m}issile & \texttt{m} \\ \hline
		zničení náhodného nepřítele & \textbf{d}elete & \texttt{d} \\ \hline
		zvýšení skóre o 100 & \textbf{s}core & \texttt{s} \\ 
		
	\end{tabular}
%	\captionsetup{textformat=empty,labelformat=blank}
	\caption{seznam cheatů}
\end{sheetfigure}
\paragraph{Pauza} 
pokud okno hry ztratí \textit{focus}, nebo hráč stiskne klávesu \texttt{p}, hra se pozastaví. Toto platí i při minimalizaci okna, což je pouze speciální případ.
\begin{figure}[H]
	\centering
	\includegraphics[scale=0.22]{paused_game.png}
	\caption{pozastavená hra}
\end{figure}
\subsubsection{Modely postav}
\label{subsub_sec:models}
V následujícím textu budu rozumět pojmem \textit{model} grafiku hry, která je vykreslená z předpřipraveného \texttt{.png} obrázku. \\ \\
Nepřátelská invaze je složena z několika různých modelů. 
\begin{figure}[H]
	\centering
	\includegraphics[scale=1]{invader_types.png}
	\caption{sloupec nepřátelské invaze}
\end{figure}

\begin{enumerate}[label=\textbf{(\arabic*)}]
	\item TopInvader -- vzhledem k jeho menšímu hitboxu je za jeho eliminaci 30 bodů.
	\item MiddleInvader -- těch je nejvíce, za eliminaci je 20 bodů.
	\item BottomInvader je první na ráně a hráč za něj získá 10 bodů.
\end{enumerate}
\noindent
Každý z těchto tří prochází při pochodu dvěma stavy pohybu.

\begin{figure}[H]
	\centering
	%[width=0.15\textwidth]
	\includegraphics{top2.png}%
	\begin{minipage}[b]{0.1\textwidth}
		\centering
		\huge$\leadsto$\normalsize
		\vspace{0.3cm}
	\end{minipage}%
	\includegraphics{top1.png}
	%\captionsetup{textformat=empty,labelformat=blank}
    \caption{stavy TopInvadera}

\end{figure}
\begin{figure}[H]
	\centering
	%[width=0.15\textwidth]
\includegraphics{middle2.png}%
\begin{minipage}[b]{0.1\textwidth}
	\centering
	\huge$\leadsto$\normalsize
	\vspace{0.3cm}
\end{minipage}%
\includegraphics{middle1.png}
%\captionsetup{textformat=empty,labelformat=blank}
\caption{stavy MiddleInvadera}
\end{figure}
\begin{figure}[H]
	\centering
	%[width=0.15\textwidth]
	\includegraphics{bottom2.png}%
	\begin{minipage}[b]{0.1\textwidth}
		\centering
		\huge$\leadsto$\normalsize
		\vspace{0.3cm}
	\end{minipage}%
	\includegraphics{bottom1.png}
%	\captionsetup{textformat=empty,labelformat=blank}
	\caption{stavy BottomInvadera}
\end{figure}
\noindent
Tito střílí po našem tanku:
\begin{figure}[H]
	\centering
	\textcolor{white}{,,}\includegraphics{lightning.png} .
	\caption{nepřátelské střelivo}
\end{figure}
\noindent
Posledním speciálním členem nepřátelské invaze je tzv. mystery. Ten s náhodnou rychlostí 3 až 10 px za 15ms, v náhodném čase z rozmezí 10 až 20s prosviští vrchní částí herní plochy. Podle jeho rychlosti jsou za jeho eliminaci přiděleny body. 
\begin{figure}[H]
	\centering
	\includegraphics{mystery.png}
	\caption{mystery}
\end{figure}
\noindent
Mystery jako jediný člen invaze neprochází mezi dvěma stavy a nestřílí. Pouze se plynule pohybuje. \\ \\
Pokud se z tanku laserem trefíme do invadera, který umí střílet zahraje znělka \ref{item:invader_elimination}. Při zásahu mysteryho zahraje \ref{item:mystery_elimination}. Mimo zvukového efektu se na pozici rozstřeleného nepřítele objeví buďto:
\begin{figure}[H]
	\centering
	\textcolor{white}{,,}\includegraphics{boom.png}\,,
	%\captionsetup{textformat=empty,labelformat=blank}
	\caption{eliminace invadera}
	\label{fig:invader_eliminated_pic}
\end{figure}
\noindent
nebo
\begin{figure}[H]
	\centering
	\includegraphics{mystery_boom.png}
%	\captionsetup{textformat=empty,labelformat=blank}
	\caption{eliminace mysteryho}
\end{figure}

\noindent
v případě eliminace mysteryho. \\ \\
Nyní se zaměřím na modely týkající se hráčova tanku.
\begin{figure}[H]
	\centering
	\begin{subfigure}{.2\textwidth}
		\centering
		\includegraphics[width=.7\linewidth]{tank.png}
		\caption{tank}
	\end{subfigure}%
	\begin{subfigure}{.16\textwidth}
		\centering
		\includegraphics[width=1.0\linewidth]{tank_destroyed.png}
		\caption{tank zasažen}
	\end{subfigure}
	%\captionsetup{textformat=empty,labelformat=blank}
	\caption{tank a jeho ruiny}
\end{figure}
%\vspace{-1cm}
\noindent
Pří zasažení tanku se v případě, že máme ještě rezervní život, spustí příslušná zvuková hláška \ref{item:tank_hit} a po dobu 300ms se zobrazí jeho ruiny (b). Následně se objeví na náhodné pozici nový tank, který bude po dobu 3s žlutý a imunní vůči nepřátelským střelám. Po uplynutí doby, kdy je tank nezranitelný, se jeho barva změní opět na zelenou. 
\begin{figure}[H]
	\centering
	\includegraphics{immune_tank.png}
	\caption{imunní tank}
\end{figure}
\noindent
V opačném případě, kdy už život nemáme, se zobrazí konec hry. \\ \\
Modely jsem vystříhal z různých variant her Space Invaders a upravil pomocí Adobe Photoshopu.
\subsubsection{Konec hry}
Hra může skončit ze dvou důvodů:
\begin{enumerate}
	\item tank byl zasažen a nezbývá už žádný rezervní život,
	\item invadeři se dostali na pozici tanku.
\end{enumerate}
\begin{figure}[H]
	\centering
	\includegraphics[scale=0.22]{gameover.png}
	\caption{GAME OVER!}
\end{figure}
\subsubsection{Pochod invaze}
Invaze vždy začíná na levé straně herní plochy a pochoduje směrem k pravé straně do té doby, než dosáhne pravého okraje. Následně invaze sestoupí blíže k tanku, zvýší svoji rychlost pochodu a obrátí směr pochodu na levou stranu. \\ \\ Invaze pochoduje periodicky a to po uplynutí času na časovači.
\subsubsection{Úrovně}
Hra začíná první úrovní. Při eliminaci všech střílecích nepřátel hra postoupí do další úrovně. Všechny úrovně mají stejné rozestavení nepřátel z invaze. Úrovně hry se liší pouze v následujícím:
\begin{itemize}
	\item[--] invaze nepřátel z vyšší úrovně začíná na pozici blíže k tanku,
	\item[--] invaze z vyšší úrovně sestupuje směrem dolů, blíže k tanku o větší vzdálenost,
	\item[--] s každým přiblížením k tanku se zrychlí pochod celé invaze. \newline (čím vyšší úroveň, tím vyšší zrychlení)
\end{itemize}
\subsubsection{Bariéry}
\label{subsub_sec:barrier}
Během vytváření nové úrovně hry se vygenerují 3 bariéry. Tyto bariéry jsou vytvořeny a vykreslovány po sloupcích ze čtverců o rozměru $5 \times 5$ px. \\ \\
Nyní bych rád popsal, jak funguje plošné poškození bariéry. \\ \\
Pokud dojde ke kolizi bariéry a nějaké střely, ať už nepřátelské, nebo naší pocházející z tanku, tak se nejprve zjistí kolikátý sloupec bariéry byl zasažen. V tomto sloupci se zničí všechny čtverečky, jejichž vzdálenost je menší než \texttt{SplashRadius}. Ve všech ostatních sloupcích se\newline s pravděpodobnostní 70\% zničí čtverečky, jejichž vzdálenost je menší než \texttt{SplashRadius} a s 30\% pravděpodobností zůstanou zachovány. Tímto vznikne efekt plošného poškození.
\begin{figure}[H]
	\centering
	%[width=0.15\textwidth]
	\includegraphics[scale=0.5]{before_splash_damage.png}%
	\begin{minipage}[b]{0.1\textwidth}
		\centering
		\huge$\leadsto$\normalsize
		\vspace{1cm}
	\end{minipage}%
	\includegraphics[scale=0.5]{after_splash_damage.png}
	\caption{před plošným poškozením a po}
\end{figure}
\noindent
Pokud se nepřátelé dostanou na hranici bariéry, všechny bariéry jsou instantně zničeny, toto zničení je avizováno hláškou \ref{item:barrier_destroyed}.
\subsubsection{Střelba}
Invadeři, jak již bylo několikrát zmíněno, střílejí po tanku. Nyní bych chtěl popsat jak funguje proces výběru invadera, který bude střílet. \\ \\
Invadeři střílejí po uplynutí času na příslušném časovači. Při vytvoření nové úrovně hry se interval časovače nastaví na 1s. Při eliminaci invadera se tento čas snižuje o 5ms. Pokud invadeři sestoupí blíže k tanku, tento interval se sníží o 30ms až do minima 420ms. \\ \\
Vždy střílí 1 až 3 invadeři v závislosti na náhodném výběru. S 30\% pravděpodobností se vybere invader, který má šanci zasáhnout tank, tj. invader, jehož souřadnice Y je nejblíže k souřadnici Y tanku. Zbylá 70\% pravděpodobnost pokrývá náhodný výběr invadera, který bude střílet. \\ \\
Pokud zbývá už pouze poslední střílecí invader nastane speciální případ. Interval na časovači určující rychlost střelby se na nastaví na 350ms a interval na časovači určující rychlost pochodu na 100ms. Invader bude velmi rychle střílet a bude obtížné ho zasáhnout.
\subsubsection{Autentičnost}
\label{subsub_sec:authenticity}
Vzhledem k tomu, že Space Invaders je už 40 let stará hra, snažil jsem se mé verzi dát alespoň jakýsi nádech grafiky původní doby. Nejvýznamnější podíl na tom má písmo Space Invaders \ref{item:space_invaders}.
\begin{figure}[H]
	\centering
	\includegraphics{space_invaders_title.png}
	\caption{nápis Space Invaders v písmu Space Invaders}
\end{figure}
\noindent
Dále jsem napodobil menu hry z původní verze z roku 1978.
\begin{figure}[H]
	\centering
	\begin{subfigure}{.5\textwidth}
		\centering
		\includegraphics[width=.7\linewidth]{1978_menu.png}
		\caption{původní menu z roku 1978}
	\end{subfigure}%
	\begin{subfigure}{.5\textwidth}
		\centering
		\includegraphics[width=.7\linewidth]{my_menu.png}
		\caption{moje menu}
	\end{subfigure}
%	\captionsetup{textformat=empty,labelformat=blank}
	\caption{porovnání původního menu z 1978 a mého menu}
\end{figure}
\noindent
A reimplementoval jsem bariéry, které jsou plošně poškozovány při střelbě.
\begin{figure}[H]
	\centering
	\begin{subfigure}{.5\textwidth}
		\centering
		\includegraphics[width=.4\linewidth]{original_barrier.png}
		\caption{původní bariéra}
	\end{subfigure}%
	\begin{subfigure}{.5\textwidth}
		\centering
		\includegraphics[width=.6\linewidth]{my_barrier.png}
		\caption{moje bariéra}
	\end{subfigure}
	%\captionsetup{textformat=empty,labelformat=blank}
	\caption{porovnání původní bariéru s mojí bariérou}
\end{figure}
%porovnani moje bariery vs 1978
\subsubsection{Responzivita}
Ačkoliv mi knihovna WinForms příliš nepomáhá, tak se hra přizpůsobí velikosti okna aplikace. Změnu velikosti okna není možné provést při spuštěné hře. Pro hraní hry v jiné velikosti okna než výchozí, musíme změnit velikost okna již v menu před započetím hry.
\begin{figure}[H]
	\centering
	\begin{subfigure}{.5\textwidth}
		\centering
		\includegraphics[width=.7\linewidth]{game_with_long_height.png}
		\caption{hra s velkou výškou}
	\end{subfigure}%
	\begin{subfigure}{.5\textwidth}
		\centering
		\includegraphics[width=1.0\linewidth]{game_with_long_width.png}
		\caption{hra s velkou sířkou}
	\end{subfigure}
	%\captionsetup{textformat=empty,labelformat=blank}
	\caption{porovnání hry s velkou délkou a šířkou}
\end{figure}
\subsubsection{Zvukové efekty}
Hra je doprovázena zvukovými efekty a hudbou. 
\paragraph{Menu}
V hlavní nabídce hraje hudba \ref{item:menu_music} z roku 1980, která pochází z australského hudebního nočního vysílání \href{https://en.wikipedia.org/wiki/Rage_(TV_program)}{Rage}.\\ \\
Další zvukové efekty:
\begin{itemize}
	\item[--] výstřel z tanku \ref{item:tank_shoot},
	\item[--] eliminace invadera  \ref{item:invader_elimination},
	\item[--] eliminace mysteryho \ref{item:mystery_elimination},
	\item[--] průlet mysteryho  \ref{item:mystery_flying},
	\item[--] zásah tanku \ref{item:tank_hit},
	\item[--] konec hry \ref{item:gameover}
\end{itemize}
pochází z různých verzí hry Space Invaders. Zbylé zvukové efekty jsem přidal dle vlastního uvážení.
\pagebreak
\subsection{Programátorská část}
Druhá část se podstatně liší od první, uživatelské části. Zde je popsána struktura projektu, jednotlivé soubory a bloky kódu, které implementují mechaniky hry. Čtenáři doporučuji, aby si nejdříve prošel část uživatelskou, neboť se tu budou často rozvíjet koncepty již částečně popsané v předchozí části.
\subsubsection{Struktura projektu}
	\begin{forest}
		for tree={font=\sffamily, grow'=0,
			folder indent=.9em, folder icons,
			edge=densely dotted}
		[SpaceInvaders
		   [SpaceInvaders, this folder size=20pt
		   [Properties
		      [(zde je několik souborů týkajících se sestavení projektu), is file]]
		   [Resources
		   [obrázky... (*.png), is file]
		   [zvuky... (*.wav), is file]
		   [space\_invaders.ttf, is file]
		   [app.ico, is file]]
		   [App.config, is file]
		   [Barrier.cs, is file]
		   [Constants.cs, is file]
		   [ExtensionMethods.cs, is file]
		   [Form1.cs, is file]
		   [Form1.Designer.cs, is file]
		   [Form1.resx, is file]
		   [Game.cs, is file]
		   [Invader.cs, is file]
		   [Program.cs, is file]
		   [ScoreTableUC.cs, is file]
		   [ScoreTableUC.Designer.cs, is file]
		   [SpaceInvaders.csproj, is file]
		   [Utils.cs, is file]
		]
		[SpaceInvaders.sln, is file]
		]
	\end{forest}
\subsubsection{Adresáře projektu}
\paragraph{SpaceInvaders/Properties}
Zde se nachází několik souborů obsahujících verzi, název, jazyk,... výsledného spustitelného souboru hry.
\paragraph{SpaceInvaders/Resources}
Tato složka obsahuje soubory, které aplikace při svém běhu využívá. Při kompilaci projektu se tyto soubory přidají do spustitelného souboru hry. \\ \\
Zbytek \hyperref[sec:program]{kapitoly 3} již bude obsahovat informace o souborech nacházejících se v adresáři \newline \textbf{SpaceInvaders/}. Budu tedy předpokládat, že se v tomto adresáři pohybujeme a jeho název již budu vynechávat.
\subsubsection{Game.cs}
Součástí souboru je jádro celé hry, které obsahuje veškerou logiku herního módu. Tento soubor je jednoznačně nejdůležitějším a největším ze všech souborů projektu. Jeho popis je rozdělen do několika částí, z nichž v každé bude rozebráno několik důležitých a ne zcela jednoduše uchopitelných celků souboru. \\ \\
V této části se vyskytuje několik seznamů, ať už proměnných nebo metod, jejichž pořadí je stejné jako jejich pořadí deklarace v programu.
\paragraph{Hlavní smyčka} Každá hra potřebuje nějakou hlavní smyčku, ve které se skokově aktualizuje herní prostředí. V této hře to má na starosti časovač \texttt{drawingTimer} s nastaveným intervalem tiknutí na 15ms. Při tiknutí se zavolá metoda \texttt{drawingTimer\_Elapsed}, kde se herní plocha překreslí a o odpovídající vzdálenost se posunou jednotlivé objekty. Dále se zde řeší veškeré kolize prvků, které mohou nastat. 
\paragraph{Vlastnosti}
\begin{itemize}
	\item[--] \texttt{Form1 Form \{ get; \}} \quad formulář aplikace nad kterým je spuštěna hra
	\item[--] \texttt{int Life \{ get; private set; \} = 5;} \quad obsahuje zbývající počet životů
	\item[--] \texttt{int Level \{ get... set... \}} \quad aktuální úroveň hry
	\item[--] \texttt{int Score \{ get... set... \}} \quad současné skóre
\end{itemize}
\paragraph{Proměnné}
\begin{itemize}
	\item[--] \texttt{List<PointF> tankMissiles} \quad kolekce střel tanku, střela je uložena bodem s největší hodnotou souřadnice Y
	\item[--] \texttt{List<PointF> explosions} \quad obsahuje pozice padlých invaderů u kterých se ještě objevuje posmrtný efekt tj. \hyperref[fig:invader_eliminated_pic]{obrázek 9}
	\item[--] \texttt{List<PointF> explosionsSafe} \quad obdobné jako bod výše, jenom z této kolekce se posmrtný efekt vykresluje, aby nedošlo k synchronizačním problémům
    \item[--] \texttt{List<PointF> invadersMissiles} \quad součástí jsou střely pocházející od invaderů, uloženy jsou bodem s nejmenší hodnotou Y
    \item[--] \texttt{List<PointF> invadersMissilesSafe} \quad obdobné jako \texttt{explosionsSafe}
    \item[--] \texttt{List<Invader> invaders} \quad seznam všech střílecích invaderů
    \item[--] \texttt{List<Barrier> barriers} \quad kolekce bariér
    \item[--] \texttt{List<RectangleF> bottomLine} \quad jednotlivé obdélníčky tvořící spodní zelenou čáru
    \item[--] \texttt{PointF? mysteryLocation} \quad pozice mysteryho, \texttt{null} pokud žádný mystery není na herní ploše
    \item[--] \texttt{PointF? mysteryExplosion} \quad obdobné jako \texttt{explosions}, pouze pro mysteryho
    \item[--] \texttt{PointF tankLocation} \quad souřadnice tanku
    \item[--] \texttt{Timer mysteryExplosionTimer} \quad časovač, který tikne když by měl posmrtný efekt mysteryho zmizet
    \item[--] \texttt{Timer drawingTimer} \quad časovač hlavní smyčky
    \item[--] \texttt{Timer mysteryShouldAppearTimer} \quad tik nastane, když by se měl objevit nový mystery, interval tohoto časovače je nastaven jako \texttt{random.Next(10000, 20000)}
    \item[--] \texttt{Timer invadersShouldMoveTimer} \quad  má na starost pochod invaze
    \item[--] \texttt{Timer explosionsTimer} \quad obdobné jako \texttt{mysteryExplosionTimer}, jenom pro střílecí nepřátele z invaze
    \item[--] \texttt{Timer invadersShouldShootTimer} \quad při tiku vystřelí 1 až 3 invadeři v závislosti\break na náhodném generátoru
    \item[--] \texttt{Timer tankExplosionTimer} \quad tikne potom, co jsme 300ms viděli ruiny našeho předchozího tanku, umístí na náhodnou pozici tank nový a dá mu na 3s imunitu
    \item[--] \texttt{Timer immunityTimer} \quad tento časovač naopak tikne, když jsme si imunitu užívali 3s\break a sebere nám ji
    \item[--] \texttt{Timer tankCanShootTimer} \quad po výstřelu tank 1s nabíjí, časovač tikne, když tank už nabil a může opět střílet
    \item[--] \texttt{bool canTankShoot} \quad \textit{flag} indikující zda tank může střílet, po výstřelu je nastaven\break na \texttt{false}, následně ho na \texttt{true} nastaví \texttt{tankCanShootTimer} po uplynutí doby nutné\break na nabití
    \item[--] \texttt{bool keyLArrow, keyRArrow, keySpace} \quad obsahují \texttt{true} pokud je daná klávesa stisknuta, v opačném případě \texttt{false}
    \item[--] \texttt{bool invadersMovingRight} \quad směr kterým pochoduje invaze
    \item[--] \texttt{bool lckMovement} \quad zámek na pohyb invaze, využívá se pokud invaze dojde na kraj herní plochy, před sestoupením invaze je totiž menší prodleva
    \item[--] \texttt{bool lockingToken} \quad udává pokud právě probíhá vykonávání kódu metody\newline \texttt{drawingTimer\_Elapsed}
    \item[--] \texttt{bool shouldDrawTank} \quad udává zda-li se má vykreslovat tank, nebo jeho ruiny
    \item[--] \texttt{bool immunity} \quad \textit{flag}, v kterém je uloženo zda je imunita tanku aktivní
    \item[--] \texttt{bool forcePause} \quad zajišťuje funkci pauzy v případě pozastavení hry hráčem v okamžiku, kdy se právě zobrazovaly ruiny tanku
    \item[--] \texttt{bool paused} \quad byla hra pozastavena hráčem stisknutím klávesy \texttt{p}?
    \item[--] \texttt{bool endGame} \quad nastal konec hry?
    \item[--] \texttt{bool enableExtraSpeed} \quad indikuje aktivaci nastavení nejvyšší rychlosti pohybu invaderů, tato hodnota se nastavuje na \texttt{true}, pokud na hracím poli zbývá pouze jeden invader
    \item[--] \texttt{bool enemyShootingDisabled} \quad udává zapnutí/vypnutí cheatu, kdy nepřátelé přestanou střílet
    \item[--] \texttt{bool barrierDestroyedPlayed} \quad nastaven na \texttt{true}, pokud byla přehrána znělka \ref{item:barrier_destroyed}, jinak je \texttt{false}
    \item[--] \texttt{int mysterySpeed} \quad rychlost prolétajícího mysteryho, jedná se o náhodně generovanou hodnotu $x$ z intervalu 3 až 10 znamenající rychlost $x$\,px za 15ms
    \item[--] \texttt{int movementAccelerationMs} \quad při sestupu invaze se interval časovače\newline \texttt{invadersShouldMoveTimer} sníží o hodnotu \texttt{movementAccelerationMs}, pochod se tedy zrychlí
    \item[--] \texttt{int greenLineY} \quad souřadnice Y spodní zelené čáry, její hodnota se spočítá na základě velikosti okna
    \item[--] \texttt{double shootingPrecision} \quad pravděpodobnost "inteligentního" výběru střelce tj. výběr střelce, který má šanci zasáhnout tank
    \item[--] \texttt{double invadersShouldMoveOriginalInterval} \quad při sestupu invaze se na chvíli změní hodnota intervalu časovače \texttt{invadersShouldMove} pro docílení menší prodlevy při sestupování invaze, poté je třeba obnovit původní hodnotu intervalu časovače, ta se uloží\break do proměnné \texttt{invadersShouldMoveOriginalInterval}
    \item[--] \texttt{float bottomInvaderY} \quad souřadnice Y invadera s největší hodnotou této souřadnice, užitečné to je pro detekci, zda se už invadeři nedostali příliš blízko a nemělo by dojít\break k instantnímu zničení bariér
    \item[--] \texttt{float barrierY} \quad souřadnice Y bariéry, její hodnota je závislá na velikosti okna
    \item[--] \texttt{float splashRadius} \quad poloměr značící maximální vzdálenost od bodu střetu střely\break s bariérou a zbytkem bariéry, kde může dojít k jejímu poškození
    \item[--] \texttt{StopWatch immunityStopwatch} \quad pokud by hra byla pozastavena během imunity tanku, nikdy by nedošlo k tiku časovače \texttt{immunityTimer}, protože při pozastavení hry se všechny časovače zakazují, to znamená, že se nám znovu resetuje imunita na 3s. Jde o nežádoucí efekt a ten řeší tyto stopky, které zaznamenají již uplynulý čas imunity, interval \newline\texttt{immunityTimer}u se o naměřený čas zkrátí
    \item[--] \texttt{Random random} \quad náhoda se vždy hodí, ať už se jedná o výběr střelce, rychlost mysteryho nebo místo, kde se po zasažení objeví nový tank
\end{itemize}
\paragraph{Konstrukce objektu Game}
Během vytváření instance třídy se v následujícím pořadí provede:
\begin{itemize}
	\item[--] inicializace časovače hlavní smyčky,
	\item[--] registrace potřebných událostí \texttt{Form1},
	\begin{itemize}
		\item \texttt{Paint} \quad vykresluje herní plochu
		\item \texttt{Resize} \quad pozastaví hru při minimalizaci okna
		\item \texttt{KeyDown} \quad nastaví stisk všech kláves na \texttt{true}
    	\item \texttt{KeyUp} \quad nastaví stisk všech kláves na \texttt{false}
		\item \texttt{Deactivate} \quad pozastaví hru při ztrátě \textit{focus}u
	\end{itemize}
    \item[--] vytvoření bariér,
    \item[--] vytvoření spodní zelené čáry, tj. čára nacházející se těsně pod tankem
    \begin{figure}[H]
    	\centering
   \textcolor{white}{,,}\includegraphics[scale=1]{green_bottom_line.png} ,
    \caption{spodní zelená čára}
     \end{figure}
    \item[--] registrace událostí všech ostatních časovačů,
    \item[--] spuštění 1. úrovně hry.
\end{itemize}
\paragraph{Uvolnění objektu Game}
Uvolnění objektu probíhá následovně:
\begin{enumerate}
	\item zastaví se všechny časovače,
	\item překreslí se okno hry,
	\item zobrazí se konec hry,
	\item uvolní se objekt \texttt{Game} zavoláním metody \texttt{Dispose}.
\end{enumerate}
\paragraph{Veřejné metody} Třída obsahuje pouze 2 veřejné metody a to sice:
\begin{itemize}
	\item[--] \texttt{CheatKey}, jenž je volána z přetížené metody \texttt{ProcessCmdKey}, metoda \texttt{CheatKey} zjistí zda byla stisknuta klávesa, která aktivuje cheat, a v případě, že ano, tak zavolá metody, které cheat aktivují,
	\item[--] \texttt{End}, ta zastaví všechny časovače, zobrazí konec hry a uvolní objekt \texttt{Game}.
\end{itemize}
\paragraph{Soukromé metody}
Oproti veřejným metodám je soukromých o poznání více. Některé\break na vstupu dostávají více parametrů a nemusí být zcela jasné k čemu slouží. Všechny metody, kde hrozí zmatení ze vstupních parametrů, je mají popsány. Z popisu je vynechán návratový typ \texttt{void}, události formuláře a události časovačů které byly popsány v rámci popisů časovačů jako proměnných.
%
\begin{itemize}
	\item[--] \texttt{GenerateBarrier()} \quad $3 \times$ zavolá metodu \texttt{CreateBarrier}, což odpovídá počtu bariér\break na herní ploše
	\item[--] \texttt{PutTextIntoControl(int value, Label l)} \quad bezpečně aktualizuje vlastnost WinForms komponent \texttt{Text} z jiného vlákna
	\item[--] \texttt{PrepareForMovingDown()} \quad zajišťuje menší prodlevu při sestupu invaze, dochází tu\break ke změně tiku časovače \texttt{invadersShouldMoveTimer} a k nastavení zámku pohybu \texttt{lckMovement}
	\item[--] \texttt{MovingDown()} \quad po zavolání metody \texttt{PrepareForMovingDown} se vykoná kód této metody, ta zajistí posun všech invaderů a úpravu intervalu časovačů
	\item[--] \texttt{Invader RightInvader()} \quad vrátí invadera z kolekce \texttt{invaders} s největší hodnotou\break souřadnice X
    \item[--] \texttt{Invader LeftInvader()} \quad oproti předchozímu bodu, tato metoda vrátí invadera s nejmenší hodnotou souřadnice X, tyto dvě metody společně se hodí k detekci zda-li se invaze\break už nedostala na kraj herní plochy
    \item[--] \texttt{CreateBottomLine()} \quad naplní proměnnou \texttt{bottomLine} obdélníčky tvořícími spodní zelenou čáru
    \item[--] \texttt{CreateBarrier(PointF origin, float w, float h, int partN, ref Barrier b)} \newline naplní prázdnou bariéru předanou referencí obdélníčky 
        \begin{enumerate}[label={\arabic*. parametr},leftmargin=0.95in]
    	\item \quad pozice levého horního rohu bariéry
    	\item \quad šířka bariéry
    	\item \quad výška bariéry
    	\item \quad bariéra se vytváří po jednotlivých částech, tento parametr určuje část,\newline\phantom{\quad}která je právě vytvářena, standardně by tato metoda měla být volána\newline\phantom{\quad}s tímto parametrem nastaveným na 1
    	\begin{figure}[H]
    		\centering
    		\includegraphics[scale=0.6]{parts_of_barrier.png}
    		\caption{části bariéry}
    	\end{figure}
    	\item \quad bariéra, která se naplní obdélníčky
    \end{enumerate}
    \item[--] \texttt{SmallShootingSpeedIncrease()} \quad tato metoda je volána po zásahu invadera pro 5ms zrychlení tiku časovače \texttt{invadersShouldShootTimer}
    \item[--] \texttt{DisableAllTimers()} \quad pomocí reflexe metoda zastaví všechy časovače
    \item[--] \texttt{bool CollisionWithBarrier(PointF missile, Barrier barrier, out int columnIdx, bool fromInvader)} \quad návratová hodnota určuje, jestli došlo ke kolizi bariéry na vstupu a \texttt{missile}
    \begin{enumerate}[label={\arabic*. parametr},leftmargin=0.95in]
    	\item \quad střela
    	\item \quad bariéra 
    	\item \quad pokud metoda vrátí \texttt{true}, tak do 3. parametru uloží index sloupce bariéry, \phantom{\quad}kde došlo ke kolizi
    	\item \quad indikátor původce střely
    \end{enumerate}
    \item[--] \texttt{SplashDamage(PointF impactPoint, Barrier barrier, int columnIdx, \newline bool fromInvader)} \quad aplikuje plošené poškození bariéry vymazáním příslušných čtverečků tvořících bariéru
    \begin{enumerate}[label={\arabic*. parametr},leftmargin=0.95in]
    	\item \quad bod bariéry, který byl zasažen střelou
    	\item \quad objekt zasažené bariéry
    	\item \quad index sloupce bariéry, v kterém se nachází \texttt{impactPoint}
    	\item \quad pochází střela zasahující bariéru od invadera?
    \end{enumerate}
    \item[--] \texttt{ShowTankExplosion()} \quad bez zobrazení zprávy o tom, že by hra byla pozastavena ji pozastaví, během té doby nastaví několik \textit{flag}ů, díky kterým se budou vykreslovat ruiny tanku, pauzu pak zruší \texttt{tankExplosionTimer}, který tato metoda aktivuje, časovač pak\break v okamžiku tiku nastaví tank imunním a zruší pozastavení
    \item[--] \texttt{NextLevel()} \quad zvýšení úrovně hry, úprava intervalů časovačů pro zvýšení obtížnosti
    \item[--] \texttt{List<Invader> NClosestInvaders(int n)} \quad vrátí seznam $N$ invaderů takových,\break aby platilo, že vzdálenost souřadnice X invadera a tanku je minimální
    \item[--] \texttt{int InvaderComparator(Invader i1, Invader i2)} \quad komparátor, díky kterému je možné seřadit kolekci invaderů podle vzdáleností souřadnice X invadera a tanku, to se hodí\break v případě, že chceme vybrat vhodné střelce
    \item[--] \texttt{MakeInvaderShootMissile(Invader invader)} \quad na vstupu dostane instanci třídy \texttt{Invader}, metoda přidá do kolekce \texttt{invadersMissiles} novou střelu s nastavenou pozicí tak, aby vypadala, jako že pochází od invadera na vstupu
    \item[--] \texttt{GenerateInvadersForLevel()} \quad naplní seznam \texttt{invaders} armádou nových invaderů
    \item[--] \texttt{UpdateBottomInvader()} \quad zaktualizuje proměnnou \texttt{bottomInvaderY}
    \item[--] \texttt{ShowNormalExplosion(PointF invaderLocation)} \quad na vstupu dostane pozici padlého invadera, metoda zajistí, aby se na této pozici zobrazila posmrtná exploze přidáním této pozice do kolekce \texttt{explosions}
    \item[--] \texttt{ShowMysteryExplosion()} \quad obdobné jako \texttt{ShowNormalExplosion}, pouze verze pro mysteryho
    \item[--] \texttt{PauseWOMsg()} \quad zastavení hry bez zobrazení nápisu "PAUSED", hodí se pro vytvoření menší prodlevy při zásahu tanku a 300ms zobrazení jeho ruin
    \item[--] \texttt{PauseGameWMsg()} \quad standardní zastavení hry vyvolané uživatelem stisknutím klávesy \texttt{p}
    \item[--] \texttt{ResumeGame()} \quad spustí všechy časovače a zajistí pokračování hry
    
\end{itemize}
\subsubsection{Form1.cs}
Jediný formulář hry, spouští se hned po startu aplikace. Během startu programu se nejprve nastaví ikonka aplikace, poté se všem \texttt{Label}ům nastaví písmo Space Invaders viz
\hyperref[codefig:setting_space_invaders_font]{Ukázka kódu 1}. Jako poslední se spustí hudba \ref{item:menu_music}, která hraje ve smyčce. \\ \\
Po startu hry se zobrazí již nám známé (z části \ref{subsub_sec:authenticity}) menu aplikace s jedním tlačítkem\break pro spuštění hry. Kód tohoto souboru obsahuje logiku pro reakci aplikace při interakci s tlačítkem. Jmenovitě se jedná o metody:
\begin{itemize}
	\item[--] \texttt{playLabel\_MouseClick} sloužící ke spuštení herního módu,
	\item[--] \texttt{clickableLabel\_MouseMove}, \texttt{clickableLabel\_MouseLeave} měnící barvu textu a vzhled kurzoru při přejetí myší.
\end{itemize}
Další metody:
\begin{itemize}
	\item[--] \texttt{SetGameMode},
	\item[--] \texttt{EnableResizing}, \texttt{DisableResizing},
	\item[--] \texttt{SetScoreAndLevelLabelVisibility},
	\item[--] \texttt{SetMenuVisibility},
	\item[--] \texttt{ShowGameOver}
\end{itemize}
upravují uživatelské rozhraní při přechodu módů (menu, hra, konec hry). \\ \\
Pokud se program nachází ve hře, potřebuje se dozvědět případné stisknuté klávesy. K tomu slouží přetížená metoda \texttt{ProcessCmdKey}, která volá instanční metodu \texttt{CheatKey}, jež náleží objektu \texttt{Game}.
\subsubsection{ScoreTableUC.cs}
Jedná se o seskupení 16 WinForms komponent na obrázku do logického celku.
\begin{figure}[H]
	\centering
	\includegraphics[scale=0.4]{score_table_uc.png}
	\caption{uživatelská komponenta ScoreTableUC}
\end{figure}
\noindent
Jedinou věcí, kterou má tento celek na starost, je nastavení písma Space Invaders všem jeho\break 12 \texttt{Label}ům. To je naštěstí v \Csharp\ primitivní:
\begin{codefigure}[H]
	\centering
\begin{BVerbatim}
foreach (Control c in Controls)
{
   if (c is Label)
      Utils.ChangeFontToSpaceInvaders(c);
}.
\end{BVerbatim}
\caption{nastavení písma Space Invaders všem \texttt{Label}ům}
\label{codefig:setting_space_invaders_font}
%\vspace{-1.4cm}
\end{codefigure}
\subsubsection{Invader.cs}
Soubor Invader.cs obsahuje hned 4 třídy. Jejich vztah je následující:
\begin{figure}[H]
\centering
\begin{tikzpicture}[every tree node/.style=draw,level 1/.style={sibling distance=10mm}]
\tikzset{level distance=50pt}
\Tree [.{\texttt{abstract Invader}}
[.{\texttt{TopInvader}} ]
[.{\texttt{MiddleInvader}} ]
[.{\texttt{BottomInvader}}
]
]
\end{tikzpicture}
%\captionsetup{textformat=empty,labelformat=blank}
\caption{vztah tříd v souboru Invader.cs}
\end{figure}
\noindent
Bázová třída obsahuje několik vlastností, z nichž většina je virtuálních. Nevirtuální jsou pouze 2 a jedná se o \texttt{Location} a \texttt{State}. První vlastnost je typu \texttt{PointF} a tedy určuje pozici invadera. Druhá je typu \texttt{bool} a její hodnota definuje stav invadera, více o stavech v části \ref{subsub_sec:models}. \\ \\
Mezi virtuální vlastnosti patří:
\begin{itemize}
	\item[--] \texttt{int ScoreGain}, což je skóre, které získáme při eliminaci invadera,
	\item[--] \texttt{Image Look1}, \texttt{Image Look2}, každá z těchto vlastností obsahuje obrázek invadera v jednom stavu,
	\item[--] \texttt{int Width}, \texttt{int Height} obsahující šířku a výšku obrázku invadera, což je v pořádku, protože všechny obrázky jednoho invadera mají stejné rozměry.
\end{itemize}
Co se týče metod, tak zde není potřeba využívat polymorfismus, protože invadeři pochodují společně a vždy se posunou o stejnou vzdálenost. Bázová třída obsahuje metody:
\begin{itemize}
	\item[--] \texttt{Image ActiveLook()}, ta vrací aktuální obrázek invadera podle jeho stavu,
	\item[--] \texttt{void SwapState()} prohazující stav invadera,
	\item[--] \texttt{void MoveLeft()}, \texttt{void MoveRight()}, obě dvě nejprve zavolají metodu \texttt{SwapState}, poté změní pozici invadera příslušným směrem o krok,
	\item[--] \texttt{void MoveDown(int lvl)}, která vyžaduje jako vstupní parameter úroveň hry, podle parametru se nastaví velikost kroku, který invader udělá.
\end{itemize}
Žádné další nezmíněné vlastnosti ani metody nejsou obsaženy v žádném z potomků bázové třídy. Samozřejmě všechny virtuální vlastnosti a metody jsou patřičně přetíženy.
\subsubsection{Barrier.cs}
Již bylo zmíněno v uživatelské části, že každá bariéra se skládá ze čtverečků o rozměru $5\times5$ vykreslovaných po sloupcích. Bude se tedy hodit čtverečky uskupit po sloupcích hlavně kvůli implementaci plošného poškození. Soubor vypadá následovně:
\begin{codefigure}[H]
	\centering
\begin{BVerbatim}
class Barrier
{
   public List<Column> Columns { get; set; } = new List<Column>();
   public float TopY { get; set; } = float.MaxValue;
}

class Column
{
   public List<RectangleF> Rectangles { get; set; } = new List<RectangleF>();
}
\end{BVerbatim}
%\captionsetup{textformat=empty,labelformat=blank}
\caption{soubor Barrier.cs}
%\vspace{-1.4cm}
\end{codefigure}
\noindent
Poslední vlastnost s zatím nejasnou funkcí je \texttt{TopY}, do ní se ukládá souřadnice Y čtverečku\break s největší hodnotou Y. To se hodí v případě, kdy by nepřátelé měli instantně zničit bariéru. \\ \\
Další informace o bariéře se nachází v části \ref{subsub_sec:barrier}.
\subsubsection{Constants.cs}
V tomto souboru můžeme nalézt několik proměnných, jejichž hodnota se nikdy nemění. Jde např. o délku střely tanku, dobu po kterou je tank imunní, rozměr čtverečků tvořících bariéry a další...
\subsubsection{ExtensionMethods.cs}
Jak již je z názvu patrné, zde se nacházejí metody, které "zvenčí" rozšiřují instanční metody třídy. Přesněji řečeno v tomto souboru se nachází pouze jedna metoda a tou je \texttt{DrawInvader}.
\begin{codefigure}[H]
\centering
\begin{BVerbatim}
public static void DrawInvader(this Graphics g, Invader invader)
{
   g.DrawImage(invader.ActiveLook(), invader.Location);
}
\end{BVerbatim}
\caption{metoda \texttt{DrawInvader}}
%\vspace{-1.4cm}
\end{codefigure}
\noindent
Na pozici invadera se vykreslí obrázek reprezentující invadera v závislosti na jeho aktuálním stavu.
\subsubsection{Utils.cs}
Pro případy, kdy využití statické třídy je přehlednější a dává smysl oproti vymýšlení \textit{extension metod}, je tu soubor Utils.cs. V něm je možné nalézt klasické metody, jako např. vzdálenost dvou bodů nebo jestli se bod nachází uvnitř obdélníku. \\ \\ O něco zajímavější je metoda \texttt{Play} \ref{item:play}, která na svém vstupu dostane \texttt{UnmanagedMemoryStream}. Tento proud dat bude reprezentovat zvuková data a ta se přehrají pomocí \texttt{SoundPlayer}u. \\ \\
Dále se zde nachází dvojce metod umožňující nastavit komponentě WinForms písmo Space Invaders, jmenovitě: \texttt{InitSpaceInvadersFont} a \texttt{ChangeFontToSpaceInvaders} viz \ref{item:font}. \\ \\ Poslední dvě metody \texttt{TopOfTankMissile} a \texttt{BottomOfInvaderMissile} mají symetrické chování. Obě na vstupu dostanou bod a vrátí bod. Nejprve se zaměřím na první metodu, ta bude\break na vstupu dostávat střely z tanku. Střela, znázorněná na následujícím obrázku, je uložena\break v paměti bodem \textbf{(1)}. Pokud metoda na vstup dostane bod \textbf{(1)}, vrátí bod \textbf{(2)}.
\begin{figure}[H]
	\centering
	\includegraphics[scale=0.2]{tank_missile.png}
	\caption{zvětšená střela z tanku}
\end{figure}
\noindent
Symetricky pro druhou metodu, střela od invadera je uložena v paměti bodem \textbf{\{1\}}. Pokud dostane na vstup bod \textbf{\{1\}}, vrátí bod \textbf{\{2\}}.
\begin{figure}[H]
	\centering
	\includegraphics[scale=0.7]{invader_missile.png}
	\caption{zvětšená střela od invadera (invertované barvy)}
\end{figure}
\noindent
\subsubsection{Další soubory}
\paragraph{App.config}
XML soubor obsahující verzi prostředí .NET Framework nutnou ke spuštění hry.
\paragraph{SpaceInvaders.csproj} Opět XML soubor, jeho součástí jsou informace o souborech přítomných v projektu Visual Studia.
\pagebreak
\section{Závěr}
Povedlo se mi implementovat hru Space Invaders v takovém rozsahu, v jakém jsem si ji představoval. GDI+ je plně dostačující pro vytvoření grafiky takovéto hry. Nejedná se ale o žádného "rychlíka". Použitím DirectX nebo OpenGL bych mohl snížit zatížení procesoru přesunutím výpočtů\break na grafickou kartu. Na druhou stranu práce s GDI+ je velmi jednoduchá. \\ \\
Dále by bylo zajimavé rozšířit hru o:
\begin{itemize}
	\item[--] různá rozestavení invaze nepřátel v závislosti na úrovni hry,
	\item[--] "padání" bonusů z padlých nepřátel (inspirace z Arkanoidu).
\end{itemize}
\newpage
\listoffigures
\listofsheetfigures
\listofcodefigures
\pagebreak
\section*{Zdroje}
    \paragraph{Grafické modely}
    Veškeré grafické modely byly vystříhány z her Space Invaders a upraveny v Adobe Photoshop.
    \paragraph{Autentičnost hry}
    Písmo Space Invaders, který celá hra používá obzvlášť zlepšil herní zážitek.
    \begin{enumerate}[label={[\arabic*]}]
    	\item space\_invaders.ttf \label{item:space_invaders}
    	\begin{enumerate}[label=]
    		\item \url{https://fonts2u.com/download/space-invaders-regular.font}
    	\end{enumerate}
    \end{enumerate}
    \paragraph{Informace o hře Space Invaders}
    Nějaké obecné informace o hře se mi hodily zejména\break pro napsání \texttt{\hyperref[sec:about_space_invaders]{kapitoly 1}} a usnadnily mi výběr zápočtového programu.
    \begin{enumerate}[label={[\arabic*]}]
  		\setcounter{enumi}{1}
    	\item anglická wikipedie
    	\begin{itemize}[label=]
    		\item \url{https://en.wikipedia.org/wiki/Space_Invaders}
    	\end{itemize}
        \item různé pohledy a ukázky ze hry na youtube
       	\begin{itemize}[label=]
       		\item \url{https://www.youtube.com/watch?v=D1jZaIPeD5w}
       		\item \url{https://www.youtube.com/watch?v=UZlEXl9xgR8}
       		\item \url{https://www.youtube.com/watch?v=MU4psw3ccUI}
       \end{itemize}
    \end{enumerate}
    \paragraph{C\# inženýrské věci}
    Během vývoje hry jsem si zaznamenával kolikrát, a co jsem sháněl jak udělat  (kde jinde než na stackoverflow \Winkey).
    \begin{enumerate}[label={[\arabic*]}]
  		\setcounter{enumi}{3}
    	\item jak použít externí font ve WinForms \label{item:font}
   	    \begin{itemize}[label=]
   	    	\item \url{https://stackoverflow.com/questions/1297264/using-custom-fonts-on-a-label-on-winforms}
      	\end{itemize}
      \item jak spustit zvukovou nahrávku \label{item:play}
      \begin{itemize}[label=]
      	\item \url{https://stackoverflow.com/questions/3502311/how-to-play-a-sound-in-c-net}
      \end{itemize}
      \item jak spojit všechny \texttt{List<T>} do jednoho pomocí LINQ
      \begin{enumerate}[label=]
      	\item \url{https://stackoverflow.com/questions/1191054/how-to-merge-a-list-of-lists-with-same-type-of-items-to-a-single-list-of-items}
      \end{enumerate}
      \item jak používat resources 
      \begin{enumerate}[label=]
      	\item \url{https://stackoverflow.com/questions/90697/how-to-create-and-use-resources-in-net}
      \end{enumerate}
      \item jak zjistit, že okno ztratilo \textit{focus}
      \begin{enumerate}[label=]
      	\item \url{https://stackoverflow.com/questions/570021/forms-lost-focus-in-c-sharp/570045}
      \end{enumerate}
      \item jak updatovat GUI z jiného threadu
      \begin{enumerate}[label=]
      	\item \url{https://stackoverflow.com/questions/10775367/cross-thread-operation-not-valid-control-textbox1-accessed-from-a-thread-othe}
      \end{enumerate}
     \pagebreak
     \item jak vytvořit cyklus přes všechny proměnné objektu
     \begin{itemize}[label=]
     	\item \url{https://stackoverflow.com/questions/9893028/c-sharp-foreach-property-in-object-is-there-a-simple-way-of-doing-this}
     	\item \url{https://stackoverflow.com/questions/7649324/c-sharp-reflection-get-field-values-from-a-simple-class}
     \end{itemize}
     \item jak zjistit, zda-li se dva obdélníky překrývají
     \begin{itemize}[label=]
     	\item \url{https://stackoverflow.com/questions/14662796/check-if-two-areas-are-in-contact}
     \end{itemize}
     \item jak spočítat počet řádek kódu celého VS projektu
     \begin{itemize}[label=]
     	\item \url{https://stackoverflow.com/questions/1244729/how-do-you-count-the-lines-of-code-in-a-visual-studio-solution}
     \end{itemize}
     \item proč se nevolá událost \texttt{KeyDown}
     \begin{itemize}[label=]
     	\item \url{https://stackoverflow.com/questions/3172731/forms-not-responding-to-keydown-events}
     \end{itemize}
    \end{enumerate}
	\paragraph{Zvukové soubory}
\begin{enumerate}[label={[\arabic*]}]
	\setcounter{enumi}{13}
	\item hudba v menu  \label{item:menu_music}
	\begin{itemize}[label=]
		\item \url{https://www.youtube.com/watch?v=8teuyCYeDxQ}
	\end{itemize}
	\item výstřel z tanku \label{item:tank_shoot}
	\begin{itemize}[label=]
		\item \url{http://www.classicgaming.cc/classics/space-invaders/files/sounds/shoot.zip}
	\end{itemize}
    \item eliminace invadera \label{item:invader_elimination}
    \begin{itemize}[label=]
    	\item \url{http://www.classicgaming.cc/classics/space-invaders/files/sounds/invaderkilled.zip}
    \end{itemize}
   \item eliminace mysteryho \label{item:mystery_elimination} 
   \begin{enumerate}[label=]
   	\item \url{http://www.classicgaming.cc/classics/space-invaders/files/sounds/ufo_highpitch.zip}
   \end{enumerate}
   \item průlet mysteryho \label{item:mystery_flying}
   \begin{enumerate}[label=]
   	\item \url{http://www.classicgaming.cc/classics/space-invaders/files/sounds/ufo_lowpitch.zip}
   \end{enumerate}
   \item zásah tanku \label{item:tank_hit}
   \begin{enumerate}[label=]
   	\item \url{http://www.classicgaming.cc/classics/space-invaders/files/sounds/explosion.zip}
   \end{enumerate}
	\item konec hry \label{item:gameover}
	\begin{itemize}[label=]
		\item \url{https://www.youtube.com/watch?v=AD0-L9L_NrM}
	\end{itemize}
    \item zničení bariéry \label{item:barrier_destroyed}
    \begin{itemize}[label=]
    	\item \url{https://uloz.to/!5878yRtYUY0C/barrier-destroyed-wav}
    \end{itemize}
\end{enumerate}
\end{document}