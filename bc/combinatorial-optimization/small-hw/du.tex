% !TeX spellcheck = cs_CZ
% !TeX program = lualatex
\documentclass[12pt, a4paper, twoside]{article}
\usepackage[czech]{babel}
\usepackage[left=1in,top=1in,right=1in,bottom=1in]{geometry}
\usepackage[utf8]{inputenc}
\usepackage{caption}
\usepackage{fancyhdr}
\usepackage{amssymb}
\usepackage{amsmath}
\usepackage{unicode-math}
\usepackage{parskip}
\setmainfont{XITS}
\setmathfont{XITS Math}

\lhead{\headerL}
\chead{\headerC}
\rhead{\headerR}
\def\headerL{}
\def\headerC{}
\def\headerR{}
\captionsetup{labelformat=empty}

\def\doubleunderline#1{\underline{\underline{#1}}}

\usepackage{tikz}
\newcommand*\circled[1]{\tikz[baseline=(char.base)]{
		\node[shape=circle,draw,inner sep=2pt] (char) {#1};}}
%\fbox{text} na ctverec kolem textu
	
\fancypagestyle{pageStyle}{%
	\fancyhf{}% Clear header/footer
	\fancyhead[OR]{Jakub Levý \footnotesize{(awk)}}% Author on Odd page, Centred
	\fancyfoot[C]{\thepage}%
	\renewcommand{\headrulewidth}{0pt}
}

\addtolength{\jot}{3pt}
	
\begin{document}
	\pagestyle{pageStyle}
	
	\section*{Zadání}
	Pomocí simplexové metody nalezněte nejprve přípustné bázické řešení a následně i optimální řešení následující úlohy:
\begin{equation*}
\begin{array}{ll@{}ll}
\text{maximalizuj}  & 4x_1+x_3+x_4 \\[8pt]
\text{za podmínek}
& 8x_1 - 5x_3- x_4&=40   \\
& 4x_2 - x_3 - x_4 &= 24 \\
& x_3  +x_5 &= 8 \\
& -2x_3+x_4+x_6 &=8 \\
& x_1,\ldots,x_6 &\ge 0 
\end{array}
\end{equation*}
\section*{Řešení}
Lineární program je zadán v rovnicovém tvaru s vektorem pravých stran $b = \begin{pmatrix} 40 \\ 24 \\  8 \\ 8 \end{pmatrix}$, $b \ge 0$. Abychom mohli simplexovou metodou najít optimální řešení, potřebujeme najít přípustné bázické řešení. To nalezneme vyřešením pomocného programu:
\begin{equation*}
\begin{array}{ll@{}ll}
\text{maximalizuj}  & -p_1-p_2-p_3-p_4 \\[8pt]
\text{za podmínek}
& 8x_1 - 5x_3- x_4 + p_1 &=40   \\
& 4x_2 - x_3 - x_4 + p_2 &= 24 \\
& x_3  +x_5 + p_3 &= 8 \\
& -2x_3+x_4+x_6 + p_4 &=8 \\
& x_1,\ldots,x_6, p_1,\ldots,p_4 &\ge 0 
\end{array}
\end{equation*}

U pomocného programu víme přípustnou bázi. Můžeme postupovat simplexovou metodou:
\begin{minipage}{0.3\textwidth}
	\begin{align*}
	p_1 &= 40 - 8x_1 + 5x_3 + x_4 \\
	p_2 &= 24 - 4x_2 + x_3 + x_4  \\
	\text{\circled{$p_3$}} &= 8-x_3-\text{\circled{$x_5$}} \\
	p_4 &= 8+2x_3 - x_4 - x_6 \\
	\makebox[0pt][l]{\underline{\phantom{$z = -80 + 8x_1 + 4x_2 - 7x_3 - x_4 + x_5 + x_6$}}}
	z &= -80 + 8x_1 + 4x_2 - 7x_3 - x_4 + x_5 + x_6 \\[14pt]	
	x_5 &= 8 - x_3 - p_3 \\
	p_1 &= 40 - 8x_1 + 5x_3 + x_4 \\
	p_2 &= 24 - 4x_2 + x_3 + x_4 \\
	\text{\circled{$p_4$}} &= 8 + 2x_3 -x_4 - \text{\circled{$x_6$}} \\
	\makebox[0pt][l]{\underline{\phantom{$z = -72 + 8x_1 + 4x_2 - 8x_3 - x_4 + x_6$}}}
	  z &= -72 + 8x_1 + 4x_2 - 8x_3 - x_4 + x_6 
	\end{align*}
\end{minipage}
\hfill\vline\hfill
\begin{minipage}{0.3\textwidth}
	\begin{align*}
		  x_6 &= 8 + 2x_3 -x_4 - p_4 \\
	x_5 &= 8 - x_3 - p_3 \\
	\text{\circled{$p_1$}} &= 40 - \text{\circled{$8x_1$}} + 5x_3 + x_4 \\
	p_2 &= 24 - 4x_2 + x_3 + x_4 \\
	\makebox[0pt][l]{\underline{\phantom{$z = -64+8x_1+4x_2-6x_3-2x_4-p_4$}}}
	z &= -64+8x_1+4x_2-6x_3-2x_4-p_4 \\[18pt]
	x_1 &= \frac{40+5x_3+x_4-p_1}{8} \\
	x_6 &= 8+2x_3 - x_4 - p_4 \\
	x_5 &= 8 -x_3 - p_3 \\
	\text{\circled{$p_2$}} &= 24 - \text{\circled{$4x_2$}} + x_3 + x_4 \\
	\makebox[0pt][l]{\underline{\phantom{$z = -24 + 4x_2 - x_3 - x_4 - p_1 - p_4$}}}
	z &= -24 + 4x_2 - x_3 - x_4 - p_1 - p_4
	\end{align*}
\end{minipage}
\newpage
\begin{align*}
x_2 &= \frac{24+x_3+x_4-p_2}{4} \\
x_6 &= 8 + 2x_3 - x_4 - p_4 \\
x_5 &= 8-x_3-p_3 \\
x_1 &= \frac{40+5x_3+x_4-p_1}{8} \\
\makebox[0pt][l]{\doubleunderline{\phantom{$z = 0 + -p_1 - p_2 - p_4$}}}
z &= 0 + -p_1 - p_2 - p_4
\end{align*}
Hodnota účelové funkce již nejde zvýšit a její hodnota je $0$, původní lineární program má optimální řešení. Z tabulky dostáváme přístupné bázické řešení původního LP $(5,6,0,0,8,8)$.
\\ \\
Odstraněním sloupců obsahující libovolnou proměnnou $p_i$ z předchozí tabulky a nahrazením účelové funkce za původní dostáváme první simplexovou tabulku původního LP:\\
\begin{minipage}{0.4\textwidth}
	\begin{align*}
	x_1 &= \frac{40+5x_3+x_4}{8} \\
	x_2 &= \frac{24+x_3+x_4}{4} \\
	\text{\circled{$x_5$}} &= 8-\text{\circled{$x_3$}} \\
	x_6 &= 8+2x_3-x_4 \\
	\makebox[0pt][l]{\underline{\phantom{$z = 20+\frac{7}{2}x_3 + \frac{3}{2}x_4$}}}
	z &= 20+\frac{7}{2}x_3 + \frac{3}{2}x_4
	\end{align*}
\end{minipage}
\hfill\vline\hfill
\begin{minipage}{0.4\textwidth}
	\begin{align*}
	x_3 &= 8 - x_5 \\
	x_1 &= \frac{80-5x_5+x_4}{8}\\
	x_2 &= \frac{32-x_5+x_4}{4}\\
	\text{\circled{$x_6$}} &= 24-2x_5 - \text{\circled{$x_4$}} \\
	\makebox[0pt][l]{\underline{\phantom{$z = 48-\frac{7}{2}x_5 + \frac{3}{2}x_4$}}}
	z &= 48-\frac{7}{2}x_5 + \frac{3}{2}x_4
	\end{align*}
\end{minipage}
\\
\phantom{c}\\
\noindent\rule{\textwidth}{1pt}
\begin{align*}
x_4 &= 24-2x_5 - x_6 \\
x_3 &= 8-x_5 \\
x_1 &= \frac{104-7x_5-x_6}{8} \\
x_2 &= \frac{56-3x_5-x_6}{4} \\
\makebox[0pt][l]{\doubleunderline{\phantom{$z = 84 - \frac{13}{2} x_5 - \frac{3}{2}x_6$}}}
z &= 84 - \frac{13}{2} x_5 - \frac{3}{2}x_6
\end{align*}
Hodnota účelové funkce již nejde zvýšit a její hodnota je 84, nalezli jsme optimální řešení. K~tomuto řešení odpovídá přípustné bázické řešení $(13,14,8,24,0,0)$.
\end{document}	