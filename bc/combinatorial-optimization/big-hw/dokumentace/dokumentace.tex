% !TeX spellcheck = cs_CZ
\documentclass[12pt, a4paper]{article}
\usepackage[czech]{babel}
\usepackage[left=1in,top=1in,right=1in,bottom=1in]{geometry}
\usepackage[utf8]{inputenc}
\usepackage{caption}
\usepackage{fancyhdr}
\usepackage{amssymb}
\usepackage{amsmath}
\usepackage{float}
\usepackage[T1]{fontenc}
\usepackage[scale=1.2]{libertine}


\renewcommand{\familydefault}{\sfdefault}
\lhead{\headerL}
\chead{\headerC}
\rhead{\headerR}
\def\headerL{}
\def\headerC{}
\def\headerR{}
\captionsetup{labelformat=empty}


\usepackage{tikz}
\newcommand*\circled[1]{\tikz[baseline=(char.base)]{
		\node[shape=circle,draw,inner sep=2pt] (char) {#1};}}
	
\fancypagestyle{firstpage}
{
	\lhead{\today}
	\chead{NOPT048}
	\rhead{Jakub Levý \footnotesize{(awk)}}
}
	
\begin{document}
	\thispagestyle{firstpage}
	\section*{Dokumentace}
	\subsection*{Struktura projektu}

	\begin{center}
	\begin{tabular}{ c|c|c } 
		
		\textbf{adresář} & \textbf{popis} \\ 
		\textit{prog/} & program generující LP úlohy 1 a ILP úlohy 2 \\
		\textit{dokumentace.pdf} & tento soubor \\ 
	   \end{tabular}
	\end{center}
Program je napsán v C++. Podle vstupu se určí výstup.
\subsection*{Kompilace programu}
Program využívá možnosti standardu C++17, je nutné používat aktuální verzi kompilátoru. Pro kompilaci stačí použít \texttt{cmake} a \texttt{make}.
\begin{figure}[H]
\texttt{[pepa@pocitadlo prog]\$ mkdir output ; cd output }\\ \\
\texttt{[pepa@pocitadlo output]\$ cmake .. } \\ \\
\texttt{[pepa@pocitadlo output]\$ make } \\ \\
\texttt{[pepa@pocitadlo output]\$ ls} \\
\texttt{CMakeCache.txt	CMakeFiles  cmake\_install.cmake  \textcolor{blue}{LP1}	Makefile}\\
\caption{příklad kompilace \textit{prog}}
\end{figure}
\subsection*{Vstup/výstup}
\begin{itemize}
	\item[--] Korektní vstup ukončený \texttt{EOF} je očekáván na standardním vstupu.
	\item[--] Výstup je vypsán na standardní výstup. 
	\item[--] Program ignoruje parametry předané při spuštění. 
\end{itemize}
\subsection*{Úloha 1}
Výstup \textit{prog} je vstup řešiče \texttt{glpsol} pokud je na vstupu neohodnocený graf. Odpovídá tomuto lineárnímu programu, kde $G = (V,E)$.
\begin{equation*}
	\begin{array}{ll@{}ll}
	\text{minimalizuj}  & y \\[8pt]
	\text{za podmínek}& x_u \le x_v - 1   &  & \forall(uv) \in E \\
					& x_u \le y & & \forall u \in V \\
					 & x_u \ge 0 & & \forall u \in V
	\end{array}
	\end{equation*}
Pokud LP má optimální řešení dle zadání se vypíše, v případě, že optimální řešení neexistuje (graf není acyklický) selže \texttt{solve;} v \texttt{glpsol}u a nevypíše se~žádný předem definovaný výstup. Součástí výstupu však je textový řetězec "LP HAS NO PRIMAL FEASIBLE SOLUTION".
\subsection*{Úloha 2}
Výstup \textit{prog} je vstup řešiče \texttt{glpsol} pokud je na vstupu ohodnocený graf. Odpovídá tomuto celočíselnému lineárnímu programu, kde $G = (V,E)$ \mbox{a~$w \colon E \rightarrow \mathbb{N}$}.
\begin{equation*}
	\begin{array}{ll@{}ll}
	\text{minimalizuj}  & \displaystyle\sum\limits_{(ij)\in E} x_{ij} \cdot w_{ij} \\[20pt]
	\text{za podmínek}& (1-x_{ij}) + (1-x_{jk}) + (1-x_{ki}) \le 2   &  & \forall(ij),(jk),(ki) \in E \\
					 & (1-x_{ij}) + (1-x_{jk}) + (1-x_{kl}) + (1-x_{li}) \le 3 & & \forall(ij),(jk),(kl),(li) \in E \\[10pt]
	& x_{ij} = \begin{cases} 1 & \text{hranu } (ij) \text{ odeberu} \\ 0 &  \text{jinak} \end{cases}  &&
	\end{array}
	\end{equation*}
Vzhledem k povaze této úlohy bude vždy existovat optimální řešení ILP.
\end{document}	