% !TeX spellcheck = cs_CZ
\documentclass[12pt, a4paper]{article}
\usepackage[czech]{babel}
\usepackage[left=1in,top=1in,right=1in,bottom=1in]{geometry}
\usepackage[utf8]{inputenc}
\usepackage{caption}
\usepackage{fancyhdr}
\usepackage{amssymb}
\usepackage{amsmath}
\usepackage{float}

\lhead{\headerL}
\chead{\headerC}
\rhead{\headerR}
\def\headerL{}
\def\headerC{}
\def\headerR{}
\captionsetup{labelformat=empty}

\usepackage{tikz}
\newcommand*\circled[1]{\tikz[baseline=(char.base)]{
		\node[shape=circle,draw,inner sep=2pt] (char) {#1};}}
	
\fancypagestyle{firstpage}
{
	\rhead{Jakub Levý \footnotesize{(Usurituri)}}
}
	
\begin{document}
	\thispagestyle{firstpage}
	\section*{Rotace}
	Mějme vektor $x = (x_0,\ldots,x_{n-1})$ a vektor $y\colon y_j = x_{(j-c) \mod n}$ (vzniklý z $x$ rotací doprava o $c$ prvků). Potom:
	\begin{align*}
	\forall j\in\{ 0,\ldots,n-1 \}\colon \left(\mathrm{DFT}(x)\right)_j = \omega_n^{j \cdot (n-c)} \cdot \left(\mathrm{DFT}(y)\right)_j
	\end{align*}
	\subsection*{Rozbor}
	$x^i$ bude značit vektor vzniklý z $x$, rotací $i$ prvků doprava. \\ \\
	Rozepišme si $\mathrm{DFT}(x)$ z definice:
    \begin{alignat*}{5}
	\left(\mathrm{DFT}(x)\right)_j &= \color{blue}{x_0 \cdot \omega_n^{j \cdot 0}} &+ x_1 \cdot \omega_n^{j \cdot 1} &+ x_2 \cdot \omega_n^{j \cdot 2} &+ \ldots &+ x_{n-1} \cdot \omega_n^{j \cdot (n-1)} \\	
	\left(\mathrm{DFT}(x^1)\right)_j &= x_{n-1} \cdot \omega_n^{j \cdot 0} &+ \color{blue}{x_0 \cdot \omega_n^{j \cdot 1}} &+x_1 \cdot \omega_n^{j \cdot 2} &+\ldots&+ x_{n-2} \cdot \omega_n^{j \cdot (n- 1)} \\
	\left(\mathrm{DFT}(x^2)\right)_j &= x_{n-2} \cdot \omega_n^{j \cdot 0} &+ x_{n-1} \cdot \omega_n^{j \cdot 1} &+ \color{blue}{x_0 \cdot \omega_n^{j \cdot 2}} &+\ldots&+ x_{n-3} \cdot \omega_n^{j\cdot (n-1)} \\
	\\
	\forall p \in \{0,\ldots,n-1\} \colon \\ \left(\mathrm{DFT}(x^{p}\right)_j &= x_{n-p} \cdot \omega_n^{j \cdot 0} &+ x_{n-p+1} \cdot \omega_n^{j \cdot 1} &+ x_{n-p+2} \cdot \omega_n^{j \cdot 2} &+\ldots&+ x_{n-p-1} \cdot \omega_n^{j \cdot (n-1)}
	\end{alignat*}\noindent
	Porozujeme, že s "`posunutím"\,\! koeficientů vektoru o jednu pozici doprava by bylo potřeba o jednu pozici doleva "`posunout"\,\! i primitivní odmocniny. Což je možné alternativně interpretovat jako posunutí o $n-1$ pozic doprava. \\ \\
	Při posunutí o $c$ pozic bychom potřebovali "`posunout"\,\! odmocniny o $n-c$ pozic. Toho je možné docílit, protože platí $\omega_n^{n + c} = \omega_n^n \cdot \omega_n^c = \omega_n^c$.
	\begin{alignat*}{2}
	\omega_n^{j\cdot(n-1)} \cdot \left(\mathrm{DFT}(x^1)\right)_j &= \omega_n^{j\cdot(n-1)} \cdot \left( x_{n-1} \cdot \omega_n^{j \cdot 0} + x_0 \cdot \omega_n^{j \cdot 1} +\ldots + x_{n-2} \cdot \omega_n^{j \cdot (n-1)} \right) &= \left(\mathrm{DFT}(x)\right)_j \\
	\omega_n^{j\cdot(n-2)} \cdot \left( \mathrm{DFT}(x^2) \right)_j &= \omega_n^{j\cdot(n-2)} \cdot \left( x_{n-2} \cdot \omega_n^{j \cdot 0} + x_{n-1} \cdot \omega_n^{j \cdot 1} + \ldots + x_{n-3} \cdot \omega_n^{j\cdot(n-1)} \right) &= \left(\mathrm{DFT}(x)\right)_j \\ \\
	\forall p \in \{0,\ldots,n-1\} \colon \\
	\omega_n^{j \cdot (n - p)} \cdot \left( \mathrm{DFT}(x^p) \right)_j &= \omega_n^{j \cdot (n - p)} \cdot \left( x_{n-p} \cdot \omega_n^{j \cdot 0} + x_{n-p+1} \cdot \omega_n^{j \cdot 1} + \ldots + x_{x-p-1} \cdot \omega_n^{j\cdot(n-1)} \right) &= \left(\mathrm{DFT}(x)\right)_j
	\end{alignat*}
\end{document}	