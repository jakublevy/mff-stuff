% !TeX spellcheck = cs_CZ
\documentclass[11pt, a4paper]{article}
\usepackage[czech]{babel}
\usepackage[left=1in,top=1in,right=1in,bottom=1in]{geometry}
\usepackage[utf8]{inputenc}
\usepackage{caption}
\usepackage{fancyhdr}
\usepackage{amssymb}
\usepackage{amsmath}

\lhead{\headerL}
\chead{\headerC}
\rhead{\headerR}
\def\headerL{}
\def\headerC{}
\def\headerR{}
\newcommand{\thispageheader}[2][R]{\expandafter\def\csname header#1\endcsname{#2}}
\captionsetup{labelformat=empty}

\usepackage{tikz}
\newcommand*\circled[1]{\tikz[baseline=(char.base)]{
		\node[shape=circle,draw,inner sep=2pt] (char) {#1};}}

\begin{document}
\pagestyle{fancy}
\thispageheader[R]{Jakub Levý \footnotesize{(Usurituri)}} \noindent
\section*{Fourierovy obrazy}
Jednotlivé obrazy jsou spočteny přímo z definice DFT. Pro výpočet užitečné jsou následující dvě lemmata:
\paragraph{Lemma (součtové):}
Nechť $n, j \in \mathbb{N}$, $j$ není dělitelné $n$. Potom:
$$ \sum_{k=0}^{n-1} \left(\omega_n^j\right)^k = 0 $$
\paragraph{Důkaz:} Tupě je možné třeba sečíst geometrickou řadu.
\begin{align*} 
\sum_{k=0}^{n-1} \left(\omega_n^j\right)^k &=  \frac{(\omega_n^j)^n - 1}{\omega_n^j - 1} = \frac{\left(\omega_n^n\right)^j - 1}{\omega_n^j - 1} = \frac{1^j - 1}{\omega_n^j - 1} = 0
\end{align*}
\paragraph{Lemma (půlící):} Nechť $n \in \mathbb{N}$. Potom:
$$ \omega_n^{n/2} = \omega_2 = -1 $$
\paragraph{Důkaz:}
$\left( \omega_n^{n/2} \right)^2 = 1$. $\omega_n^{n/2} = -1$ nebo $\omega_n^{n/2} = 1$. Přicemž druhý případ není možný, protože $\omega_n$ je primitivní n-tá odmocnina z $1$.
\\ \\
V následujících příkladech \circled{1.} -- \circled{3.} předpokládejme následující:
\begin{itemize}
	\item $|x| = n$
	\item $y$ je Fourierův obraz $x$
	\item $|y| = n$
\end{itemize}
\subsection*{\circled{1.} $ x =(1,0,1,0,\ldots,1,0)$}
\paragraph{Pozorování:} $n$ je sudé. \\
Přímo z definice máme: $\displaystyle y_j = \sum_{k=0}^{n-1} x_k \cdot \omega_n^{kj}$. \newline Rozepsáním sumy dostaneme: $y_j = x_0 \cdot \omega_n^{0j} + x_2 \cdot \omega_n^{2j} + x_4 \cdot \omega_n^{4j} + \ldots + x_{n-2} \cdot \omega_n^{(n-2) j}$. \newline
Ekvivalentně: $\displaystyle y_j = \sum_{k=0}^{n/2 - 1} \omega_n^{2kj} = \sum_{k=0}^{n/2 - 1} \left( \omega_n^{2j} \right)^k$. Nyní můžeme použít součtové lemma. \newline Zjistíme, že suma je rovna $0$ pro všechna $j$ vyjma dvou případů: $j = 0$ a $j = n/2$. Na první případ součtové lemma nefunguje, protože $j$ není přirozené číslo. V druhém případě je $2j$ dělitelné $n$, což porušuje předpoklad lemmatu. \\ \\
Víme tedy, že: $y = (y_0, \underbrace{0, \ldots, 0}_{\#n/2 - 1}, y_{n/2}, \underbrace{0, \ldots,0}_{\#n/2 - 1})$. Zbývá ručně dopočítat $y_0$ a $y_{n/2}$.
\begin{align*}
y_0 &= \sum_{k=0}^{n/2 - 1} \omega_n^0 = \frac{n}{2} \\
y_{n/2} &= \sum_{k=0}^{n/2 - 1} \omega_n^{2 (n/2) \cdot k} = \sum_{k=0}^{n/2 - 1} (\omega_n^{n})^k = \sum_{k=0}^{n/2 - 1} 1^k = \frac{n}{2}
\end{align*}
Dostáváme, že Fourierův obraz $x$ je: $y = (n/2, \underbrace{0, \ldots, 0}_{\#n/2 - 1}, n/2, \underbrace{0, \ldots,0}_{\#n/2 - 1})$.
\subsection*{\circled{2.} $ x =(1,-1,1,\ldots,1,-1)$}
\paragraph{Pozorování:} $n$ je sudé. \\
Z definice víme, že: $\displaystyle y_j = \sum_{k=0}^{n-1} x_k \cdot \omega_n^{kj}$. \newline 
Hned je vidět, že $\displaystyle y_j = \sum_{k=0}^{n-1} (-1)^k \omega_n^{kj}$. \newline
Rozepsáním sumy dostaneme: $\displaystyle y_j = \omega_n^{0j} - \omega_n^{j} + \omega_n^{2j} - \omega_n^{3j} + \ldots - \omega_n^{(n-1)j}$. \newline
Seskupením sudých a lichých mocnin:
\begin{align*}
y_j &= \omega_n^{0j} + \omega_n^{2j} + \omega_n^{4j} + \ldots + \omega_n^{(n-2)j} - (\omega_n^{j} + \omega_n^{3j} + \omega_n^{5j} + \ldots + \omega_n^{(n-1)j}) \\
y_j &= \overbrace{\underbrace{\sum_{k=0}^{n/2 - 1} \omega_n^{2kj}}}^{y_{j,1}}_{\circled{1.}} - \overbrace{\sum_{k=0}^{n/2-1} \omega_n^{(2k+1)j}}^{y_{j,2}} \\
y_j &= y_{j,1} + y_{j,2}
\end{align*}
Po zbytek příkladu nás bude zajímat pouze druhá suma. \newline
\thispagestyle{plain}
Upravíme si sumu do tvaru, abychom mohli použít součtové lemma. \newline
$\displaystyle \sum_{k=0}^{n/2-1} \omega_n^{(2k+1)j} = \sum_{k=0}^{n/2-1} \left(\omega_n^{2j + j/k}\right)^{k}$ \\ \\
Případ, kdy $j = 0$ vyřešíme později zvlášť. Narozdíl od příkladu \circled{1.}, zde není tak jednoduše vidět, kdy $2j+j/k$ je dělitelné $n$. \\ \\
$\cfrac{2j + \frac{j}{k}}{n} = \cfrac{\frac{2jk}{k} + \frac{j}{k}}{n} = \cfrac{2jk+j}{kn} = \cfrac{j(2k+1)}{kn}$ \\ \\
Z toho tvaru již vidíme, že $2j+j/k$ je dělitelné $n$ pouze pro $j = n/2$. \newline
Získali jsme dva případy ($j = 0$ a $j = n/2$), ve kterých nevím hodnotu $y_{j,2}$. Ve všech ostatních případech ze součtového lemmatu platí, že $y_{j,2} = 0$.
\begin{align*}
y_{n/2,2} &= \sum_{k=0}^{n/2-1} \omega_n^{(2k+1) \cdot (n/2)} = \sum_{k=0}^{n/2-1} \omega_n^{kn + n/2} = \sum_{k=0}^{n/2 - 1} \omega_n^{kn} \cdot \omega_n^{n/2} = \sum_{k=0}^{n/2 - 1} \left(\omega_n^{n}\right)^{k} \cdot \omega_2 = - \frac{n}{2} \\
y_{0, 2} &= \sum_{k=0}^{n/2 - 1} \omega_n^0 = \frac{n}{2}
\end{align*}
Teď už známe:
\begin{align*}
y_{j,2} &= (n/2, \underbrace{0, \ldots, 0}_{\#n/2 - 1}, -n/2, \underbrace{0, \ldots, 0}_{\#n/2-1}) \\
y_{j,1} &= (n/2, \underbrace{0, \ldots, 0}_{\#n/2 - 1}, \ \: \: n/2, \underbrace{0, \ldots,0}_{\#n/2 - 1}) \quad \text{(z předchozího příkladu)}
\end{align*}
Přičemž $y_j = y_{j,1} - y_{j,2}$. \newline
Fourierův obraz $x$ je: $y = (\underbrace{0, \ldots, 0}_{\#n/2}, n, \underbrace{0, \ldots, 0}_{\#n/2 - 1})$.
\subsection*{\circled{3.} $ x =(\omega_n^0, \omega_n^1, \ldots, \omega_n^{n-1})$, \normalfont{kde} $\omega_n = \mathrm{e}^{2\pi i/n}$}
\thispagestyle{plain}
\paragraph{Pozorování:} $n$ je sudé nebo liché. \\
Z definice: $\displaystyle y_j = \sum_{k=0}^{n-1} x_k \cdot \omega_n^{kj} = \sum_{k=0}^{n-1} \omega_n^k \cdot \omega_n^{kj} = \sum_{k=0}^{n-1} \left(\omega_n^{1+j}\right)^k$. \\ \\
Okamžitě je vidět, že $1+j$ je dělitelné $n$ pokud $j = n - 1$. Pro všechna ostatní $j$ je suma rovna $0$. Zbývá tedy spočítat kolik je $y_{n-1}$ a je hotovo.
\begin{align*}
y_{n-1} = \sum_{k=0}^{n-1} \left(\omega_n^{1+n-1}\right)^k = \sum_{k=0}^{n-1} \left( \omega_n^n \right)^k = \sum_{k=0}^{n-1} 1^k = n
\end{align*}
Fourierův obraz vektoru $x$ je: $y = (\underbrace{0, \ldots, 0}_{\#n-1}, n)$.
\end{document}