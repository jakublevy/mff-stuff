% !TeX spellcheck = cs_CZ
\documentclass[12pt, a4paper]{article}
\usepackage[czech]{babel}
\usepackage[left=1in,top=1in,right=1in,bottom=1in]{geometry}
\usepackage[utf8]{inputenc}
\usepackage{caption}
\usepackage{fancyhdr}
\usepackage{amssymb}
\usepackage{amsmath}
\usepackage{complexity}
\usepackage{enumitem}

\lhead{\headerL}
\chead{\headerC}
\rhead{\headerR}
\def\headerL{}
\def\headerC{}
\def\headerR{}
\captionsetup{labelformat=empty}

\usepackage{tikz}
\newcommand*\circled[1]{\tikz[baseline=(char.base)]{
		\node[shape=circle,draw,inner sep=2pt] (char) {#1};}}
	
\fancypagestyle{firstpage}
{
	\rhead{Jakub Levý \footnotesize{(Usurituri)}}
}

\newclass{\CNF}{CNF}
	
\begin{document}
	\thispagestyle{firstpage}
	\section*{$k$-barevnost}
	Problém si rozdělíme na 2 části. Víme, že 2-$\SAT$ $\in \P$. V první části budeme chtít dokázat\\[2pt]
	\fbox{\begin{minipage}{9.5em}	
		$2$-barevnost $\rightarrow$ $2$-$\SAT$
	\end{minipage}}\\[2pt]
	Tím ukážeme, že problém 2-barevnosti $\in \P$.
	\\[12pt]
	Nechť $k \ge 3$, potom je $k$-$\SAT$ $\NP$-úplný, v druhé části budeme chtít dokázat\\[2pt]
	\fbox{\begin{minipage}{9.5em}	
			$k$-$\SAT$ $\rightarrow$ $k$-barevnost
	\end{minipage}}\\[2pt]
	Čímž získáme, že $k$-barevnost $\in \NP$.
	\subsection*{$2$-barevnost $\rightarrow$ $2$-$\SAT$}
	Na vstupu jsme dostali graf $G = (V, E)$, chceme vyrobit formuli v $\CNF$, jejíž každá klauzule obsahuje nejvýše 2 literály a je splnitelná právě tehdy když $G$ je 2-obarvitelný.
	Předpokládejme, že $V = \{ v_1, v_2,\ldots, v_n \}$. Vytvoříme si proměnné  \[v_{i,b} = 
	\begin{cases}
	1 & \text{pokud vrchol } v_i \text{ je obarven barvou }b \in \{1, 2\} \\
	0 & \text{jinak} 
	\end{cases}
	\]
	Formuli, kterou hledáme bude obsahovat následující klauzule:
	\begin{itemize}
		\item $\forall i \in \{1,\ldots,n \} \colon$
		\begin{itemize}[label={}]
			\item $(v_{i,1} \vee v_{i,2})$ \quad (každý vrchol je obarvenou alespoň 1 barvou)
			\item $\neg(v_{i,1} \wedge v_{i,2}) = (\neg v_{i,1} \vee \neg v_{i,2})$ \quad (žádný vrchol není obarven 2 barvami)
		\end{itemize}
	\item $\forall i, j \in \{1,\ldots,n \}$ takové, že $\{v_i, v_j\} \in E \colon$
	\begin{itemize}[label={}]
		\item $\neg(v_{i,1} \wedge v_{j,1}) = (\neg v_{i,1} \vee \neg v_{j,1})$ \quad \newline (vrcholy spojené hranou nejsou zároveň obarveny 1. barvou)
		\item $\neg(v_{i,2} \wedge v_{j,2}) = (\neg v_{i,2} \vee \neg v_{j,2})$ \quad \newline (vrcholy spojené hranou nejsou zároveň obarveny 2. barvou)
	\end{itemize}
	\end{itemize}
Formule v $\CNF$ obsahující tyto klauzule je splnitelná právě tehdy když graf $G$ je \newline $2$-obarvitelný. Je potřeba však ještě ukázat, že takovýto převod z grafu na formuli stihneme v polynomiálním čase. A to stihneme, protože počet klauzulí je závislý na počtu vrcholů a hran grafu, přičemž hran má graf nejvýše kvadraticky mnoho k počtu vrcholů. 
\subsection*{$k$-$\SAT$ $\rightarrow$ $k$-barevnost}
\end{document}