% !TeX spellcheck = cs_CZ
\documentclass[11pt, a4paper]{article}
\usepackage[czech]{babel}
\usepackage[left=1in,top=1in,right=1in,bottom=1in]{geometry}
\usepackage[utf8]{inputenc}
\usepackage{caption}
\usepackage{fancyhdr}
\usepackage{enumitem}
\usepackage[linesnumbered,algoruled,boxed,lined]{algorithm2e}
\SetKwInput{KwIn}{Vstup}
\SetKwInput{KwResult}{Výstup}
\SetAlgorithmName{Algoritmus}



\lhead{\headerL}
\chead{\headerC}
\rhead{\headerR}
\def\headerL{}
\def\headerC{}
\def\headerR{}
\newcommand{\thispageheader}[2][R]{\expandafter\def\csname header#1\endcsname{#2}}
\captionsetup{labelformat=empty}
\SetKwComment{Comment}{$\triangleright$\ }{}
\begin{document}
\pagenumbering{gobble}
\pagestyle{fancy}
\thispageheader[R]{Jakub Levý \footnotesize{(Usurituri)}} \noindent
\section*{Minimální vrcholové pokrytí}
K nalezení minimálního vrcholového pokrytí v bipartitním grafu se bude hodit algoritmus pro nalezení maximálního párování v bipartitním grafu a Kőnigova-Egerváryho věta z přednášek z Kombinatoriky a grafů I.
\subsection*{Značení}
$m(G)\ ... $ velikost maximálního párování v grafu $G$ \\
$v_c(G)\ ... $ velikost minimálního vrchovolého pokrytí grafu $G$
\subsection*{Rozbor}
Mějme tedy na vstupu bipartitní graf $G = (X \cup Y, E)$ s partitami $X$ a $Y$. Vytvoříme z něho síť $(G', s, t ,c)$, kde $G' = (X' \cup Y', E')$.
\begin{itemize}
	\item  $X' = X \cup \{s\}$ 
	\item $Y' = Y \cup \{t\}$.
	\item $E' = \{ (u,v) \ |\ u \in X,\; v \in Y,\; \{u,v\} \in E \} \cup \{ (s, u ) \ | \ u \in X \, \} \cup \{(u,t)\ |\ u \in Y\}$
	\begin{itemize}[label={}]
		\item hrany v síti jsou orientovány směrem "`zleva doprava", tj. od partity $X$ k $Y$
		\item od zdroje $s$ vedou hrany do každého vrcholu partity $X$
		\item od každého vrcholu partity $Y$ vede hrana do stoku $t$
	\end{itemize}
	\item $\forall e \in E' : c(e) = 1$
\end{itemize}
Nechť $f$ je maximální tok sítě $(G', s, t,c)$, potom $|f| = m(G)$ a hrany $ e\in E$, takové, že $f(e) = 1$ tvoří maximální párování grafu $G$. \\ \\
Že hrany $e \in E : f(e) = 1$ tvoří párování lze nahlédnou následovně: Žádné dvě hrany takové hrany nemůžou mít společný vrchol. Pokud by totiž společný vrchol náležel do partity $Y$, pak by do tohoto vrcholu musely přitéci alespoň 2 jednotky toku, které nemají kudy odtéct. Analogicky pro partitu $X$. \\ \\
Maximalita párování plyne z toho, že pro každý tok sítě $(G', s, t, c)$ je možné nalézt párování stejné velikosti a naopak. Existuje tedy bijekce mezi množinou všech toků a množinou všech párování, které zachovává velikost. Maximální tok tedy odpovídá maximálnímu párování. \\ \\
Teď bychom potřebovali převést problém nalezení maximálního párování a problém hledání minimálního vrcholového pokrytí. Z kurzu Kombinatoriky a grafů I známe Kőnigova-Egerváryho větu: $\forall$ bipartitní graf $G : m(G) = v_c(G)$. \\ \\ 
V následujícím odstavci bude znamenat pojem \textit{párování} maximální párování a \textit{vrcholové pokrytí} minimální vrcholové pokrytí. \\ \\
Jako vrcholy vrcholového pokrytí budeme brát vrcholy hran párování. Minimálního počtu vrcholů na vrcholové pokrytí grafu dosáhneme tak, že z každé hrany párování použijeme právě jeden její libovolný vrchol, čímž pokryjeme celou hranu. Po pokrytí všech hran párování, kterých je $m(G)$, jsme vrcholově pokryli celý graf, protože párování bylo maximální a tedy žádné další disjunktní hrany se v grafu nacházet nemohou.
\newpage
\thispagestyle{empty}
\subsection*{Pseudocode}
\begin{algorithm}[H]
	\KwIn{Bipartitní graf $G = (X \cup Y, E)$ s partitami $X$ a $Y$}
	\KwResult{Množina vrcholů tvořících minimální vrcholové pokrytí }
    Vytvoříme síť $(G', s, t, c)$ viz Rozbor\\
    $f \gets$ maximální tok sítě $(G',s,t,c)$ (F-F)\\
    $vysledek \gets \{\}$\\
    \For{$\{u,v\} \in E, u \in X, v \in Y$}{
    	\If{$f((u,v)) = 1$}{
    		
    		$vysledek$.insert(vrchol $u$, nebo $v$)
    	}
    }
	\Return{$vysledek$}
	\caption{Minimální vrcholové pokrytí}
\end{algorithm} 
\noindent
\\ \\ \\ 
Označme $n$ počet vrcholů a $m$ počet hran tokové sítě.
\begin{itemize}
	\item $n = |X| + |Y| + 2$
	\item $m = |E| + |X| + |Y|$
	\item $|V| = |X| + |Y|$.
\end{itemize}
\subsection*{Časová složitost}
Tokovou síť lze postavit v lineárním čase v počtem vrcholů a hran, tj. $\mathcal{O}(n + m)$. F-F algoritmem nalezneme maximální párování v čase $\mathcal{O}(n \cdot m)$. Poté v lineárním čase s počtem hran projdeme síť, získáme minimální vrcholové pokrytí. Celková časová složitost je $\mathcal{O}(n \cdot m) = \mathcal{O}(|V| \cdot |E|)$.
\subsection*{Prostorová složitost}
Pro uložení:
\begin{itemize}
	\item grafu tokové sítě $G'$ budeme potřebovat $\Theta(n + m)$ paměti
	\item kapacity pro každou hranu budeme potřebovat $\Theta(m)$ paměti
	\item toku $f$ také $\Theta(m)$
	\item proměnné \textit{vysledek} $\mathcal{O}(v_c(G)) \in \mathcal{O}(n)$
\end{itemize}
Výsledně tedy paměťová složitost je $\mathcal{O}(n + m) = \mathcal{O}(|V| + |E|)$.
\end{document}