% !TeX spellcheck = cs_CZ
\documentclass[12pt, a4paper]{article}
%\usepackage[czech]{babel}
\usepackage[left=1in,top=1in,right=1in,bottom=1in]{geometry}
\usepackage[utf8]{inputenc}
\usepackage{caption}
\usepackage{fancyhdr}
\usepackage{amssymb}
\usepackage{amsmath}
\usepackage{float}
\usepackage{tikz}
\usetikzlibrary{arrows,shapes.gates.logic.US,shapes.gates.logic.IEC,calc}
\usepackage{graphicx}
\graphicspath{{images/}}
\usepackage{pdfpages}

\lhead{\headerL}
\chead{\headerC}
\rhead{\headerR}
\def\headerL{}
\def\headerC{}
\def\headerR{}
\captionsetup{labelformat=empty}

\newcommand*\circled[1]{\tikz[baseline=(char.base)]{
		\node[shape=circle,draw,inner sep=2pt] (char) {#1};}}
	
\fancypagestyle{firstpage}
{
	\rhead{Jakub Levý \footnotesize{(Usurituri)}}
}
	
\begin{document}
	\thispagestyle{firstpage}
	\section*{Všemocný NAND}
	$\mathrm{XOR}(a,b) = (a \wedge  \neg b) \vee (\neg a \wedge b) \\
	\mathrm{NAND}(a,b) = \neg(a \wedge b) = \neg a \vee \neg b$ \\ \\
    Negace pomocí NANDu jde jednoduše zkonstruovat. \\
    $\mathrm{NOT}(a) = \mathrm{NAND}(a,a) = \neg a \vee \neg a = \neg a$ \\ \\
	Triviální je konstrukce XORu pomocí 5 NANDů. \\
	Označme $\varphi = (a \wedge \neg b)$, $\psi = (\neg a \wedge b)$ (první a druhá elementární konjunkce funkce XOR).
	\begin{flalign*}
	\neg \varphi &= (\neg a \vee b) = \mathrm{NAND}(a, \mathrm{NAND}(b,b))\\
	\neg \psi &= (a \vee \neg b) = \mathrm{NAND}(\mathrm{NAND}(a, a), b)\\
	\mathrm{XOR}(a, b) &= \mathrm{NAND}(\neg \varphi, \neg \psi)
	\end{flalign*}
\begin{figure}[H]
	\centering
	\includegraphics[scale=0.75]{5gate.pdf}
\end{figure}\noindent
Problém je v plýtvání hradly pro každou negaci. Koneckonců $\varphi$ a $\psi$ jsou symetrické formule.
\paragraph{Pozorování: (trik)}
\begin{flalign*}
\overbrace{\neg \left(\overbrace{(\neg a \vee \neg b)}^{\mathrm{NAND}(a,b)} \wedge a \right)}^{\mathrm{NAND}(\mathrm{NAND}(a,b),a)} &= (\neg a \vee b) = \neg \varphi \\
\underbrace{\neg \left(\underbrace{(\neg a \vee \neg b)}_{\mathrm{NAND}(a,b)} \wedge b \right)}_{\mathrm{NAND}(\mathrm{NAND}(a,b),b)} &= (a \vee \neg b) = \neg \psi
\end{flalign*}
Jeden NAND se dá šikovně ušetřit tak, že místo jednotlivých negací si spočítáme $\mathrm{NAND}(a,b)$ a využijeme triku. Na výsledný $\mathrm{XOR}(a,b)$ je použito poslední zbývající 4. hradlo stejně jako v předchozí 5 hradlové konstrukci 5. hradlo.
\begin{figure}[H]
	\centering
	\includegraphics[scale=0.75]{4gate.pdf}
\end{figure}\noindent
Každou booleovskou formuli můžeme přeložit na booleovský obvod využívající hradla $\mathrm{AND}$, $\mathrm{OR}$ a $\mathrm{NOT}$, které jsou dohromady funkčně kompletní. Zbývá tedy ukázat že z hradla $\mathrm{NAND}$ dovedeme vyrobit hradlo $\mathrm{AND}$ a $\mathrm{OR}$. Hradlo $\mathrm{NOT}$ již bylo ukázáno.
\begin{flalign*}
\mathrm{NAND}(a, b) &= ( \neg a \vee \neg b )\\
\neg(\neg a \vee \neg b) &= (a \wedge b) = \mathrm{NAND}\left(\mathrm{NAND}(a,b), \mathrm{NAND}(a,b)\right) = \mathrm{AND}(a,b)\\
\end{flalign*}
\vspace*{-2cm}
\begin{figure}[H]
	\centering
	\includegraphics[scale=0.75]{and.pdf}
\end{figure}\noindent
\begin{flalign*}
\mathrm{NAND}(a, b) &= ( \neg a \vee \neg b )\\
a \vee b &= \mathrm{NAND}\left(\mathrm{NAND}(a,a), \mathrm{NAND}(b,b)\right)
\end{flalign*}
\vspace*{-1.2cm}
\begin{figure}[H]
	\centering
	\includegraphics[scale=0.68]{or.pdf}
\end{figure}

\end{document}	