% !TeX spellcheck = cs_CZ
\documentclass[12pt, a4paper, twoside]{article}
\usepackage[czech]{babel}
\usepackage[left=1in,top=1in,right=1in,bottom=1in]{geometry}
\usepackage[utf8]{inputenc}
\usepackage{caption}
\usepackage{fancyhdr}
\usepackage{enumitem}
\usepackage{float}
\usepackage{diagbox}
\usepackage{graphicx}
\graphicspath{{images/}}
\usepackage{subfig}
\usepackage{pgfplots}
\usepackage{cancel}
\usepackage{emptypage}
\usepackage{hyperref} 
\usepackage{tikzsymbols}
\usepackage{amsmath, amssymb, graphics, setspace}
\newcommand{\mathsym}[1]{{}}
\newcommand{\unicode}[1]{{}}
\usepgfplotslibrary{fillbetween}

\lhead{\headerL}
\chead{\headerC}
\rhead{\headerR}
\def\headerL{}
\def\headerC{}
\def\headerR{}
%\captionsetup{labelformat=empty}

\newcommand{\wideunderline}[2][2em]{%
	\underline{\makebox[\ifdim\width>#1\width\else#1\fi]{#2}}%
}



\usepackage{tikz}
\newcommand*\circled[1]{\tikz[baseline=(char.base)]{
		\node[shape=circle,draw,inner sep=2pt] (char) {#1};}}
	
\newcommand*\squared[1]{\tikz[baseline=(char.base)]{
		\node[shape=rectangle,draw,inner sep=3pt] (char) {#1};}}	

\fancypagestyle{firstpage}
{
	
	\rhead{Jakub Levý}
	\chead{Domácí úkol z NMAI059}
	\lhead{list $1$ z $6$}
}

\def\checkmark{\tikz\fill[scale=0.4](0,.35) -- (.25,0) -- (1,.7) -- (.25,.15) -- cycle;} 

	
\fancypagestyle{otherpages}
{
	
	
	\rhead{Jakub Levý}
	\chead{Domácí úkol z NMAI059}
	\lhead{list $\pgfmathparse{\thepage / 2 }\pgfmathprintnumber[precision=0]{\pgfmathresult}$ z $6$}
}

\usepackage{mathtools}
\usepackage{ragged2e}
\newlength\ubwidth
\newcommand\parunderbrace[2]{\settowidth\ubwidth{$f#1$}\underbrace{#1}_{\parbox{\ubwidth}{\scriptsize\RaggedRight#2}}}
	
\begin{document}
	\thispagestyle{firstpage}
	\section*{Příklad \circled{1}}
	Mějme 3 druhy zákusků: $\square$, $\triangle$, $\heartsuit$.
	\subsection*{\squared{a}}
	Označme:
	\begin{itemize}[label={}]
		\item $A$ = ["`nemůže vybrat ze všech zákusků"]
		\item $A^c$ = ["`může vybrat ze všech zákusků"]
	\end{itemize}
Bude jednodušší spočítat $P(A^c)$ a poté vyjádřit $P(A)$ jako $1 - P(A^c)$. \\ \\
Aby si Adam mohl vybrat ze všech druhů zákusků, musí stůl se zákusky vypadat jako libovolný řádek tabulky. 
\begin{figure}[H]
	\centering
\begin{tabular}{ c  c  c }
	$\square$ & $\triangle$ & $\heartsuit \heartsuit$ \\ \hline
    $\square$ & $\triangle \triangle$ & $\heartsuit$ \\ \hline
    $\square \square$ & $\triangle$ & $\heartsuit$ 
\end{tabular}
\caption{stůl se zákusky}
\end{figure} \noindent
Označme $B_i$ = ["`na stole zbyly zákusky i-tého řádku tabulky"]. \\ Platí, že:
\begin{align*}
P(B_1) &= P(B_2) = P(B_3) = \parunderbrace{\binom{5}{2,2,1}}{\#pořadí odebírání zákusků} \quad \cdot \quad \parunderbrace{\frac{3}{9} \cdot \frac{2}{8}}{odebrání dvou zákusků 1. typu} \quad \cdot \quad \parunderbrace{\frac{3}{7} \cdot \frac{2}{6}}{odebrání dvou zákusků 2. typu} \quad \cdot \quad \parunderbrace{\frac{3}{5}}{odebrání jednoho zákusku 3. typu} = \frac{3}{14} \\
P(A^c) &= P(B_1 \cup B_2 \cup B_3) = P(B_1) + P(B_2) + P(B_3) = \frac{9}{14} \\[8pt]
P(A) &= 1 - P(A^c) = 1 - \frac{9}{14} = \underline{\phantom{d}\frac{5}{14}\phantom{d}}
\end{align*} 
\subsection*{\squared{b}}
6 lidí si vezme náhodný zákusek, aby zůstaly na stole všechny tři druhy zákusků, musí na stole zbýt od každého druhu právě jeden kus. \\
Označme $C$ = ["`na stole zůstaly všechny tři druhy zákusků"].
\begin{align*}
P(C) = \parunderbrace{\binom{6}{2,2,2}}{\#pořadí odebírání zákusků} \quad \cdot \quad \parunderbrace{\frac{3}{9} \cdot \frac{2}{8}}{odebírání dvou zákusků 1. typu} \quad \cdot \quad \parunderbrace{\frac{3}{7} \cdot \frac{2}{6}}{odebírání dvou zákusků 2. typu} \quad \cdot \quad \parunderbrace{\frac{3}{5} \cdot \frac{2}{4}}{odebírání dvou zákusků 3. typu} = \underline{\phantom{d}\frac{9}{28}\phantom{d}}
\end{align*}
\subsection*{\squared{c}}
Bez újmy na obecnosti považujme $\square$ za věneček. Po odebrání 5 zákusků bude stůl vypadat jako jeden řádek tabulky.
 \pagestyle{otherpages}
\begin{figure}[H]
	\centering
\begin{tabular}{ c  c  c }
	$\square$ & & $\heartsuit \heartsuit \heartsuit$ \\ \hline
	$\square$ & $\triangle \triangle \triangle$ & \\ \hline
	$\square$ & $\triangle$ & $\heartsuit \heartsuit$ \\ \hline 
	$\square$ & $\triangle \triangle$ & $\heartsuit$ 
\end{tabular}
\caption{stůl se zákusky}
\end{figure}\noindent
Na stole vždy leží 4 zbylé zákusky z nichž 1 je věneček, aby tam věneček zůstal, tak si Adam musí vzít libovolné jiný. \\
Označme D = ["`věneček zbyl i po Adamově výběru"], potom $P(D) = \underline{\phantom{d}\frac{3}{4}\phantom{d}}$
\section*{Příklad \circled{2}}
\subsection*{\squared{a}}
$(X,Y)^T$ je diskrétní náhodný vektor. Sdružené a marginální rozdělení je popsáno tabulkou.

\begin{figure}[H]
	\centering
	\bgroup
	\def\arraystretch{1.3}
\begin{tabular}{ c  c  c  c | c}
	\backslashbox{$X$}{$Y$} & $0$ & $1$ & $2$ \\ \hline
	$0$ & \multicolumn{1}{|c|}{$\frac{21}{32} \cdot \frac{20}{31}$}  & \multicolumn{1}{|c|}{$\binom{2}{1} \cdot \frac{3}{32} \cdot \frac{21}{31}$} & \multicolumn{1}{c|}{$\frac{3}{32} \cdot \frac{2}{31}$} & $\frac{69}{124}$ \\ \hline
	$1$ & \multicolumn{1}{|c|}{$\binom{2}{1} \cdot \frac{7}{32} \cdot \frac{21}{31}$} & \multicolumn{1}{|c|}{$\binom{2}{1} \cdot \frac{7}{32} \cdot \frac{3}{31} + \binom{2}{1} \cdot \frac{1}{32} \cdot \frac{21}{31}$} & \multicolumn{1}{c|}{$\binom{2}{1} \cdot \frac{1}{32} \cdot \frac{3}{31}$} & $\frac{12}{31}$ \\ \hline
	$2$ & \multicolumn{1}{|c|}{$\frac{7}{32} \cdot \frac{6}{31}$} & \multicolumn{1}{|c|}{$\binom{2}{1} \cdot \frac{1}{32} \cdot \frac{7}{31}$} & \multicolumn{1}{c|}{$0$} & $\frac{7}{124}$ \\ \hline
	& \multicolumn{1}{|c|}{$\frac{189}{248}$} & \multicolumn{1}{c}{$\frac{7}{31}$} & \multicolumn{1}{|c|}{$\frac{3}{248}$} & $\textbf{1}$
\end{tabular}
	\egroup
	\caption{rozdělení před zjednodušením výpočtů}
\end{figure}
\begin{figure}[H]
	\centering
	\bgroup
	\def\arraystretch{1.3}
	\begin{tabular}{ c  c  c  c | c}
		\backslashbox{$X$}{$Y$} & $0$ & $1$ & $2$ \\ \hline
		$0$ & \multicolumn{1}{|c}{$\frac{105}{248}$}  & \multicolumn{1}{|c|}{$\frac{63}{496}$} & \multicolumn{1}{c|}{$\frac{3}{496}$} & $\frac{69}{124}$ \\ \hline
		$1$ & \multicolumn{1}{|c}{$\frac{147}{496}$} & \multicolumn{1}{|c|}{$\frac{21}{248}$} & \multicolumn{1}{c|}{$\frac{3}{496}$} & $\frac{12}{31}$ \\ \hline
		$2$ & \multicolumn{1}{|c}{$\frac{21}{496}$} & \multicolumn{1}{|c|}{$\frac{7}{496}$} & \multicolumn{1}{c|}{$0$} & $\frac{7}{124}$ \\ \hline
		& \multicolumn{1}{|c|}{$\frac{189}{248}$} & \multicolumn{1}{c}{$\frac{7}{31}$} & \multicolumn{1}{|c|}{$\frac{3}{248}$} & $\textbf{1}$
	\end{tabular}
	\egroup
	\caption{rozdělení po zjednodušení výpočtů}
\end{figure}
\subsection*{\squared{b}}
Náhodné veličiny $X, Y$ jsou závislé. To je hned vidět z druhého řádku tabulky. \newline $0 = P(X = 2 \ \wedge \ Y = 2) \ne P(X = 2) \cdot P(Y = 2) = \frac{7}{124} \cdot \frac{3}{248}$ \\
O korelaci náhodných veličin z jejich závislosti nic nevíme. Bude třeba ji spočítat ručně.
\begin{align*}
EX &= 0 \cdot \frac{69}{124} + 1 \cdot \frac{12}{31} + 2 \cdot \frac{7}{124} = \frac{1}{2} \\[6pt]
EY &= 0 \cdot \frac{189}{248} + 1 \cdot \frac{7}{31} + 2 \cdot \frac{3}{248} = \frac{1}{4} \\[6pt]
EX^2 &= 0 \cdot \frac{69}{124} + 1 \cdot \frac{12}{31} + 4 \cdot \frac{7}{124} = \frac{19}{31} \\[6pt]
EY^2 &= 0 \cdot \frac{189}{248} + 1 \cdot \frac{7}{31} + 4 \cdot \frac{3}{248} = \frac{17}{62} \\[6pt]
EXY &= 1 \cdot 1 \cdot \frac{21}{248} + 1 \cdot 2 \cdot \frac{3}{496} + 2 \cdot 1 \cdot \frac{7}{496} = \frac{1}{8} \\[6pt]
\mathrm{var}\, X &= \frac{19}{31} - \frac{1}{4} = \frac{45}{124} \\[6pt]
\mathrm{var}\, Y &= \frac{17}{62} - \frac{1}{16} = \frac{105}{496} \\[6pt]
\mathrm{cov}(X, Y) &= \frac{1}{8} - \frac{1}{2} \cdot \frac{1}{4} = 0 \\[6pt]
\mathrm{corr}(X, Y) &= \frac{0}{\sqrt{\frac{45}{124} \cdot \frac{105}{496}}} = \underline{\phantom{d}0\phantom{d}}
\end{align*}
\subsection*{\squared{c}}
Náhodná veličina $Z$ může nabývat pouze hodnot $0, 1, 2$.
\begin{align*}
P(Z = 0) &= \frac{24-a}{24} \cdot \frac{23 - a}{23} \\[6pt]
P(Z = 1) &= \binom{2}{1} \cdot \frac{a}{24} \cdot \frac{24 - a}{23} \\[6pt]
P(Z = 2) &= \frac{a}{24} \cdot \frac{a - 1}{23} \\[6pt]
EZ &= 0 \cdot \frac{24-a}{24} \cdot \frac{23 - a}{23} + 1\cdot \binom{2}{1} \cdot \frac{a}{24} \cdot \frac{24 - a}{23} + 2\cdot \frac{a}{24} \cdot \frac{a - 1}{23} = \underline{\phantom{d}\frac{a}{12}\phantom{d}}
\end{align*}
\begin{figure}[H]
\centering
\bgroup
\def\arraystretch{1.3}
	\begin{tabular}{c | c c c c c c c c c}
		$a$ & $0$ & $1$ & $2$ & $3$ & $4$ & $5$ & $6$ & $7$ & $8$ \\ \hline
		$EZ$ & $0$ & $\frac{1}{12}$ & $\frac{1}{6}$ & $\frac{1}{4}$ & $\frac{1}{3}$ & $\frac{5}{12}$ & $\frac{1}{2}$ & $\frac{7}{12}$ & $\frac{2}{3}$ \\
	\end{tabular}
\egroup
\caption{$EZ$ pro všechny hodnoty parametru $a$}
\end{figure}
\subsection*{\squared{d}}
Známe střední hodnotu náhodné veličiny $Z$, která závisí na parametru $a$. Po provedení náhodného výběru $Z_1,\ldots, Z_n$ můžeme spočítat $\overline{Z_n}$. O výběrovém průměru víme, že je nestranný a konzistentní odhad střední hodnoty. Parametr $a$ je možné odhadnout z~následujícího vztahu:
\begin{align*}
EZ &= \frac{a}{12} \approx \frac{1}{n} \cdot \sum_{k=1}^{n} Z_k = \overline{Z_n}
\end{align*}
Jedná se tedy o odhad: $\widehat{a_n} = 12 \cdot \overline{Z_n}$
\paragraph{Nestranost:} 
\begin{align*}
E\widehat{a_n} = E\left(12 \cdot \overline{Z_n}\right) = 12 \cdot E\overline{Z_n} = 12 \cdot EZ = 12 \cdot \frac{a}{12} = a
\end{align*}
\paragraph{Konzistence:}\mbox{}\\
Předpoklady ZVČ jsou splněny:
\begin{itemize}[label={}]
	\item $Z_1,\ldots Z_n$ (iid) \checkmark
	\item $-\infty < EZ_i = EZ < \infty$ \checkmark
	\item $-\infty < \mathrm{var}\, EZ_i = \mathrm{var}\, EZ = \cfrac{a \cdot (22+a)}{276} - \left(\cfrac{a}{12}\right)^2 = \cfrac{264\, a - 11\, a^2}{3312} < \infty$ \checkmark
\end{itemize}
Ze ZVČ víme, že $\overline{Z_n} \xrightarrow[n \to \infty]{P} EZ$. \\[6pt] Využijeme věty o spojité transformaci.
\begin{align*}
g(t) &= 12 \,t \quad \text{(spojitá na } \mathbb{R}\text{)} \\
g(\overline{Z_n}) &= 12 \cdot \overline{Z_n}\,\, = \widehat{a_n} \\
g(EZ) &= 12 \cdot EZ = a \\[12pt]
\widehat{a_n} &\xrightarrow[n \to \infty]{P} a
\end{align*}
$\widehat{a_n}$ je nestranný a konzistentní odhad $a$.
\subsection*{\squared{e}}
Řešeno ve Wolfram Mathematice 11.3.
\begin{figure}[H]
	\begin{doublespace}
		\noindent\(\pmb{\text{(* } K - \text{kule}}\\
		\pmb{ L - \text{listy}}\\
		\pmb{ S - \text{srdce}}\\
		\pmb{ Z - \text{zaludy} }\\
		\pmb{\text{*)}}\\
		\pmb{\text{deck}=\text{RandomSample}[\text{Join}[\text{ConstantArray}[\text{\lq \lq S\rq \rq},8],\text{ConstantArray}[\text{\lq \lq Z\rq \rq},8],}\\
		\pmb{\text{\quad \quad \, \, \! \! \text{ConstantArray}[\text{\lq \lq L\rq \rq},8]}, \text{ConstantArray}[\text{\lq \lq K\rq \rq},8]]]}\)
	\end{doublespace}
	
	\begin{doublespace}
		\noindent\(\pmb{\text{(*} \text{ na } \text{n{\' a}hodn{\' y}ch } \text{indexech } \text{smazeme } \text{karty } \text{z } \text{bal{\' \i}{\v c}ku} }\\
		\pmb{\text{*)}}\\
		\pmb{\text{For}[i=1,i\leq 8,i\text{++},\text{deck}=\text{Delete}[\text{deck},\text{RandomInteger}[\{1,\text{Length}[\text{deck}]\}]]]}\)
	\end{doublespace}
	
	\begin{doublespace}
		\noindent\(\pmb{\text{(*} \text{ skute{\v c}n{\' y} } \text{po{\v c}et } \text{zb{\' y}vaj{\' \i}c{\' \i}ch } \text{srdcov{\' y}ch } \text{karet}}\\
		\pmb{\text{*)}}\\
		\pmb{\text{srdcovych}=\text{Count}[\text{deck}, \text{\lq \lq S\rq \rq}];}\)
	\end{doublespace}
	
	\begin{doublespace}
		\noindent\(\pmb{n=20;}\)
	\end{doublespace}
	
	\begin{doublespace}
		\noindent\(\pmb{Z=\text{Table}[\text{Count}[\text{Table}[\text{deck}[[\text{RandomInteger}[\{1,\text{Length}[\text{deck}]\}]]],\{i,1,2\}],\text{\lq \lq S\rq\rq}],\{k,1,n\}]}\)
	\end{doublespace}
	
	\begin{doublespace}
		\noindent\(\pmb{\text{EZ}=\frac{a}{12};}\\
		\pmb{\text{Average}=\frac{1}{n}*\sum _{k=1}^n Z[[k]];}\)
	\end{doublespace}
	
	\begin{doublespace}
		\noindent\(\pmb{\text{NSolve}[\text{EZ}==\text{Average},a]\text{ //}N}\)
	\end{doublespace}
	
	\begin{doublespace}
		\noindent\(\pmb{\text{srdcovych}}\)
	\end{doublespace}
	\caption{zdrojový kód}
\end{figure}	
\begin{figure}[H]
	\centering
	\bgroup
	\def\arraystretch{1.3}
	\begin{tabular}{c | c || c | c}
		\multicolumn{2}{c||}{$n = 20$} & \multicolumn{2}{c}{$n = 2000$} \\ \hline
		$a$ & $\hat{a}$ & $a$ & $\hat{a}$ \\ \hline
		$6$ & $5.4$ & $7$ & $6.918$ \\
		$7$ & $6$ & $6$ & $5.682$ \\
     	$6$ & $4.2$ & $6$ & $6.078$ \\
     	$7$ & $5.4$ & $7$ & $6.924$ \\
     	$5$ & $4.2$ & $6$ & $6.204$ \\
		$6$ & $7.2$ & $5$ & $4.968$ \\
		$5$ & $3.6$ & $6$ & $6.132$ \\
		$6$ & $4.2$ & $6$ & $6.048$ \\
		$5$ & $3$ & $7$ & $6.738$ \\
		$8$ & $6$ & $7$ & $7.164$
	\end{tabular}
	\egroup
	\caption{realizace pokusů}
\end{figure}\noindent 
S rostoucí velikostí náhodného výběru se přesnost odhadu zlepšuje. Pro $n=2000$ se jedná už o solidní odhad.
\section*{Příklad \circled{3}}
Označme $X$ = ["`doba výpočtu úlohy s náhodným vstupem"].
\[f_X(x) =  \begin{cases} 
\frac{2}{x^3} & x\ge 1 \\
0 & x < 1
\end{cases}
\]
\subsection*{\squared{a}}
$\displaystyle P(X < 5) = P(X > 1 \ \wedge \ X < 5) = \int_{1}^{5} f_X(x)\,dx = \int_{1}^{5} \frac{2}{x^3}\,dx = 2 \cdot \left[ \frac{1}{-2x^2} \right]_1^5 = \newline= 2 \cdot \left(\frac{1}{-50} + \frac{1}{2}\right) = \underline{\phantom{d}\frac{24}{25}\phantom{d}}$
\begin{figure}[H]
	\centering
\begin{tikzpicture}
\begin{axis}[xmax=6,ymax=2.25,ymin=-0,xmin=-1, samples=50, axis y line=middle,axis x line=bottom,xlabel={$X$}, ylabel={$f_X(X)$}, xlabel style={below right}]
\addplot[blue, name path=f, ultra thick, domain=1:6] (x,2/x^3);
\addplot[blue,  ultra thick, domain=-1:1] (x,0);
\node[label={180:{}},circle,fill,inner sep=2pt, color=blue] at (axis cs:1,2) {};
\node[label={180:{}},circle,draw,inner sep=2pt, color=blue] at (axis cs:1,0) {};

\path[name path=axis] (axis cs:0,0) -- (axis cs:6,0);


\addplot [
thick,
color=blue,
fill=blue, 
fill opacity=0.15
]
fill between[
of=f and axis,
soft clip={domain=1:6},
];
\end{axis}
\end{tikzpicture}
\caption{Graf PDF náhodné veličiny X}
\end{figure}\noindent
$P(X < 5)$ není nic jiného než pouze plocha mezi funkcemi $f_X$ a osou $X$ na \mbox{intervalu $\left(1,5\right)$.}
\subsection*{\squared{b}}
$\displaystyle EX = \int_{-\infty}^{\infty} x \cdot f_X(x)\,dx = \int_{1}^{\infty} x \cdot \frac{2}{x^3}\,dx = \lim\limits_{b\to\infty} \int_{1}^{b} \frac{2}{x^2}\,dx = \lim\limits_{b\to\infty} 2\cdot \left[ -\frac{1}{x} \right]_1^b =\\[6pt]= \lim\limits_{b\to\infty} 2\cdot \left(-\frac{1}{b} + 1\right) = \underline{\phantom{d}2\phantom{d}}$
\subsection*{\squared{c}}
$\displaystyle EX^2 = \int_{-\infty}^{\infty} x^2 \cdot f_X(x)\,dx = \int_{1}^{\infty} x^2 \cdot \frac{2}{x^3}\,dx = \lim\limits_{b\to\infty} \int_{1}^{b} \frac{2}{x}\,dx = \lim\limits_{b\to\infty} 2 \cdot \left[ 2\ln{x} \right]_1^b =\\[6pt]= \lim\limits_{b\to\infty} 2 \cdot \left( 2\ln{b} - 2 \ln{1} \right) = \infty$ \\ \\ \\
$\displaystyle \mathrm{var}\,X = EX^2 - \left(EX\right)^2 = \infty - 4 = \underline{\phantom{d}\infty\phantom{d}}$
\paragraph{Poznámka:} Všiml jsem si, že PDF $f_X$ je speciální případ \texttt{\href{https://en.wikipedia.org/wiki/Pareto_distribution}{Paretova rozdělení}}.
\[
g(x) = 
\begin{cases}
\cfrac{\alpha \cdot x_0^\alpha}{x^{\alpha + 1}} & x \ge x_0 \\[8pt]
0 & x < x_0
\end{cases}
\]
pro parametry $\alpha = 2$, $x_0 = 1$. Nechť $Y$ je náhodná veličina, která je rozdělena Paretovým rozdělením. Platí:
\[ E(Y) = 
\begin{cases}
\infty & \alpha \le 1 \\[8pt]
\cfrac{\alpha \cdot x_0}{\alpha - 1} & \alpha > 1
\end{cases}
\]
\[ \mathrm{var}\,(Y) = 
\begin{cases}
\infty & \quad \alpha \in (1,2] \\[14pt]
\left(\cfrac{x_0}{\alpha - 1}\right)^2 \cdot \cfrac{\alpha}{\alpha - 2} & \quad \alpha > 2
\end{cases}
\]
Což odpovídá předchozím výpočtům $EX$ a $\mathrm{var}\,X$.
\subsection*{\squared{d}}
Nejprve spočítáme $F_X$.
\[
F_X(t) =  
\begin{cases}
\int_{1}^{t} 2/x^3\,dx & \quad t \ge 1 \\[6pt]
\int_{-\infty}^t 0\,dt = 0 & \quad t < 1
\end{cases}
\] \\ \\
$\displaystyle \int_{1}^{t} \frac{2}{x^3} = 2 \cdot \left[ - \frac{1}{2x^2} \right]_1^t = 2 \cdot \left(- \frac{1}{2t^2} + \frac{1}{2} \right) = 1 - \frac{1}{t^2}$\\ \\
\[
F_X(x) = 
\begin{cases}
1 - \cfrac{1}{x^2} & \quad x \ge 1 \\[8pt]
0 & \quad x < 1
\end{cases}
\]
Teď bychom ze znalosti distribuční funkce chtěli generovat náhodné veličiny. Nechť:
\begin{itemize}
	\item $U \sim R[0,1]$
	\item $H\colon [0,1] \to \mathbb{R}$
	\begin{itemize}[label={}]
          \item $H(U) = X$
          \item $H$ je neklesající a invertovatelná funkce
	\end{itemize}
\end{itemize}
\begin{align*}
\forall x \in \mathbb{R}\colon F_X(x) = P(X \le x) = P(H(U) \le x) &= P\left[H^{-1}(H(U)) \le H^{-1}(x)\right] \\ &= P(U \le H^{-1}(x)) = H^{-1}(x)
\end{align*}
Dostáváme, že $\forall y \in [0,1]\colon H(y) = F_X^{-1}(y)$. Stačí tedy dostat $U$, spočítat $F_X^{-1}(U) = X$ a~je hotovo. Chybí nám však zatím ta inverzní distribuční funkce \Winkey. \\
Z distribuční funkce víme, že $X \ge 1$. Budeme hledat inverzí funkci k $\displaystyle 1 - \frac{1}{x^2}$
\begin{align*}
y &=1-\cfrac{1}{\left(F_X^{-1}(y) \right)^2} \\[6pt]
1-y &= \cfrac{1}{\left(F_X^{-1}(y) \right)^2} \\[6pt]
\cfrac{1}{1-y} &= \left(F_X^{-1}(y) \right)^2 \\[6pt]
F_X^{-1}(y) &= \cfrac{1}{\sqrt{1-y}} \quad \quad y \in [0,1) \\[6pt]
\end{align*}
\paragraph{Poznámka:}
Případ $\cancel{F_X^{-1}(y) = - \cfrac{1}{\sqrt{1-y}} \quad \quad y \in [0,1)}$ \quad nás nezajímá. $F_X^{-1}$ vrací náhodné veličiny, o kterých víme, že jsou alespoň $1$. \\ \\
Pro $y = 1$ dodefinujeme $F_X^{-1}(y) = \infty$. Což dává logický smysl, vzhledem k tomu, že $F_X^{-1}(y)$ vrací takové $x$, že $P(-\infty < X < x) \le y$. \\ \\
Finálně:
\[ F_X^{-1}(y) =
\begin{cases}
\cfrac{1}{\sqrt{1-y}} & y \in [0,1) \\[12pt]
\infty & y = 1
\end{cases}
\]
\subsection*{\squared{e}}
Řešeno ve Wolfram Mathematice 11.3.
\begin{figure}[H]
\begin{doublespace}
	\noindent\(\pmb{\text{Indicator}[\text{Condition$\_$}]\text{:=}\text{If}[\text{Condition},1,0];}\)
\end{doublespace}

\begin{doublespace}
	\noindent\(\pmb{F[\text{x$\_$}]\text{:=}\text{Piecewise}\left[\left\{\left\{1-\frac{1}{x^2},x\geq 1\right\},\{0,x<1\}\right\}\right];}\)
\end{doublespace}

\begin{doublespace}
	\noindent\(\pmb{f[\text{x$\_$}]\text{:=}\text{Piecewise}\left[\left\{\left\{\frac{2}{x^3},x\geq 1\right\},\{0,x<1\}\right\}\right];}\)
\end{doublespace}

\begin{doublespace}
	\noindent\(\pmb{\text{FInv}[\text{y$\_$}]\text{:=}\text{Piecewise}\left[\left\{\left\{\frac{1}{\sqrt{1-y}},y\geq 0\land y<1\right\},\{\infty ,y==1\}\right\}\right];}\)
\end{doublespace}

\begin{doublespace}
	\noindent\(\pmb{n=1000; \text{(*} \text{ nastavit } 20, 100, 1000\text{ }\text{ *)}}\)
\end{doublespace}

\begin{doublespace}
	\noindent\(\pmb{X=\text{Table}[\text{FInv}[\text{RandomReal}[]],\{i,1,n\}];}\)
\end{doublespace}

\begin{doublespace}
	\noindent\(\pmb{\text{FEmpirical}[\text{x$\_$}]\text{:=}\frac{1}{n}*\sum _{k=1}^n \text{Indicator}[X[[k]]\leq x]}\)
\end{doublespace}

\begin{doublespace}
	\shorthandoff{"}
	\noindent\(\pmb{\text{Plot}[\{F[x],\text{FEmpirical}[x]\},\{x,0,5\},\text{AxesLabel}\to \{\text{Style}[x,18],\text{None}\},}\\
	\pmb{\text{           }\text{BaseStyle}\to \{\text{FontSize}\to 20\},\text{PlotLegends}\to \left\{\text{Style}\left[\texttt{"}F_X\text{(x)$\texttt{"}$},18\right],\right.}\\
	\pmb{\left.\text{      }\text{Style}\left[\text{OverHat}\left[\texttt{"}F_X\texttt{"}\right]\text{{``}(x){''}},18\right]\right\},}\\
	\pmb{\text{           }\text{PlotLabel}\to \text{StringForm}[\text{{``}Porovnání CDF a ECDF (n = \`{}\`{}){''}},n]]}\)
	\shorthandon{"}
\end{doublespace}

\begin{doublespace}
	\noindent\(\pmb{\text{Average}=\frac{1}{n}*\sum _{k=1}^n X[[k]]}\)
\end{doublespace}
\caption{zdrojový kód}
\end{figure}\noindent
\begin{figure}[H]
\centering
\bgroup
\def\arraystretch{1.3}
\begin{tabular}{c |  c | c | c |}
	& \multicolumn{3}{c|}{$\overline{X_n}$} \\ \hline
	$n $ & $20$ & $100$ & $1000$ \\ \hline
	& 1.53243 & 1.8398  & 2.15269 \\ \hline		
	& 1.59385 & 2.11342 & 1.98973 \\ \hline
	& 1.88965 & 2.4047 & 2.106 \\ \hline
	& 3.07 & 2.2107 & 1.97948 \\ \hline
	& 1.69749 & 2.13945 & 2.26897 \\ \hline
	& 1.7707 & 1.88066 & 1.89993 \\ \hline
	& 2.05292 & 1.98504 & 1.9335 \\ \hline
	& 1.66008 & 1.64479 & 1.93232 \\ \hline
	& 1.9351 & 1.98355 & 1.88608 \\ \hline
	& 2.13355 & 1.88512 & 1.91333 
\end{tabular}
\egroup
\caption{realizace výběrových průměrů $X_1,\ldots,X_n$}
\end{figure}\noindent
Považujme jednotlivé výběrové průměry za náhodné veličiny $Y_{i,j}$ (indexujme jako matici).
\begin{figure}[H]
	\centering
	\bgroup
	\def\arraystretch{1.3}
	\begin{tabular}{c | c | c}
		$\overline{Y_{*,1}}$ & $\overline{Y_{*,2}}$ & $\overline{Y_{*,3}}$ \\ \hline
		2.12161 & 2.00872 & 2.00620
	\end{tabular}
	\egroup	
\end{figure}\noindent
Pozorujeme, že s rostoucí velikostí náhodného výběru se výběrový průměr náhodných veličin blíží ke střední hodnotě náhodné veličiny, ta je $2$.
\newpage
\begin{figure}[H]
	\centering
	\includegraphics[scale=0.65]{n=20.png}
\end{figure}
\begin{figure}[H]
	\centering
	\includegraphics[scale=0.65]{n=100.png}
\end{figure}
\begin{figure}[H]
	\centering
	\includegraphics[scale=0.65]{n=1000.png}
\end{figure}\noindent
Na obrázcích porovnávající CDF a ECDF je patrná kvalita (nestranost a konzistence) odhadu CDF pomocí ECDF. S dostanečně velkým náhodným výběrem ($n$ = 1000) se už jedná o velmi přesný odhad.
\end{document}	